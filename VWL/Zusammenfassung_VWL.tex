\documentclass[10pt]{scrartcl}

%Math
\usepackage{amsmath}
\usepackage{amsfonts}
\usepackage{amssymb}
\usepackage{amsthm}
\usepackage{ulem}
\usepackage{stmaryrd} %f\UTF{00FC}r Blitz!

%PageStyle
\usepackage[ngerman]{babel} % deutsche Silbentrennung
\usepackage[utf8]{inputenc} 
\usepackage{fancyhdr, graphicx}
\usepackage[scaled=0.92]{helvet}
\usepackage{enumitem}
\usepackage{parskip}
\usepackage[a4paper,top=2cm]{geometry}
\setlength{\textwidth}{17cm}
\setlength{\oddsidemargin}{-0.5cm}
\usepackage[scaled=0.92]{helvet}
\usepackage{lastpage} % for getting last page number
\renewcommand{\familydefault}{\sfdefault}
\usepackage{wrapfig}


% Shortcommands
\newcommand{\Bold}[1]{\textbf{#1}} %Boldface
\newcommand{\Kursiv}[1]{\textit{#1}} %Italic
\newcommand{\T}[1]{\text{#1}} %Textmode
\newcommand{\Nicht}[1]{\T{\sout{$ #1 $}}} %Streicht Shit durch

%Arrows
\newcommand{\lra}{\leftrightarrow} 
\newcommand{\ra}{\rightarrow}
\newcommand{\la}{\leftarrow}
\newcommand{\lral}{\longleftrightarrow}
\newcommand{\ral}{\longrightarrow}
\newcommand{\lal}{\longleftarrow}
\newcommand{\Lra}{\Leftrightarrow}
\newcommand{\Ra}{\Rightarrow}
\newcommand{\La}{\Leftarrow}
\newcommand{\Lral}{\Longleftrightarrow}
\newcommand{\Ral}{\Longrightarrow}
\newcommand{\Lal}{\Longleftarrow}

% Code listenings
\usepackage{color}
\usepackage{xcolor}
\usepackage{listings}
\usepackage{caption}
\DeclareCaptionFont{white}{\color{white}}
\DeclareCaptionFormat{listing}{\colorbox{gray}{\parbox{\textwidth}{#1#2#3}}}
\captionsetup[lstlisting]{format=listing,labelfont=white,textfont=white}
\lstdefinestyle{JavaStyle}{
 language=Java,
 basicstyle=\footnotesize\ttfamily, % Standardschrift
 numbers=left,               % Ort der Zeilennummern
 numberstyle=\tiny,          % Stil der Zeilennummern
 stepnumber=5,              % Abstand zwischen den Zeilennummern
 numbersep=5pt,              % Abstand der Nummern zum Text
 tabsize=2,                  % Groesse von Tabs
 extendedchars=true,         %
 breaklines=true,            % Zeilen werden Umgebrochen
 frame=b,         
 %commentstyle=\itshape\color{LightLime}, Was isch das? O_o
 %keywordstyle=\bfseries\color{DarkPurple}, und das O_o
 basicstyle=\footnotesize\ttfamily,
 stringstyle=\color[RGB]{42,0,255}\ttfamily, % Farbe der String
 keywordstyle=\color[RGB]{127,0,85}\ttfamily, % Farbe der Keywords
 commentstyle=\color[RGB]{63,127,95}\ttfamily, % Farbe des Kommentars
 showspaces=false,           % Leerzeichen anzeigen ?
 showtabs=false,             % Tabs anzeigen ?
 xleftmargin=17pt,
 framexleftmargin=17pt,
 framexrightmargin=5pt,
 framexbottommargin=4pt,
 showstringspaces=false      % Leerzeichen in Strings anzeigen ?        
}

 
\fancypagestyle{firststyle}{ %Style of the first page
\fancyhf{}
\fancyheadoffset[L]{0.6cm}
\lhead{
\includegraphics[scale=0.8]{./img/fhnw_logo_en.png}}
\renewcommand{\headrulewidth}{0pt}
\lfoot{school of business,\linebreak www.fhnw.ch }
}

\fancypagestyle{documentstyle}{ %Style of the rest of the document
\fancyhf{}
\fancyheadoffset[L]{0.6cm}
\lhead{
\includegraphics[scale=0.8]{./img/fhnw_logo_en.png}}
\renewcommand{\headrulewidth}{0pt}
\lfoot{\thepage\ / \pageref{LastPage} }
}

\pagestyle{firststyle} %different look of first page
 
\title{ %Titel
Volkswirtschaftslehre
\vspace{0.2cm}
\Large (Zusammenfassung)}

 \begin{document}
 \maketitle
 \thispagestyle{firststyle}
 \pagestyle{firststyle}
 \begin{abstract}
 \begin{center}
  Zusammenfassung des Modul VWL (HS14) von Fabian Stebler. \\
"THE BEER-WARE LICENSE" (Revision 42):\\
fabian.stebler@students.fhnw.ch wrote this file. As long as you retain this notice you
can do whatever you want with this stuff. If we meet some day, and you think
this stuff is worth it, you can buy me a beer in return.
 \end{center}
 \vspace{0.5cm}
\hrulefill
\end{abstract}

 \pagestyle{documentstyle}
 \tableofcontents
 \pagebreak
\section{Grundlagen}
\subsection{Begriffe}
\subsubsection{Volkswirtschaftslehre}
Die Volkswirtschaftslehre oder Ökonomie ist die Wissenschaft vom Einsatz knapper Ressourcen zur Produktion wertvoller Wirtschaftsgüter durch die Gesellschaft und von der Verteilung dieser Güter in der Gesellschaft.
\subsubsection{Effizienz \& Produktionseffizienz}
{\bf Effizienz} bedeutet, die Ressourcen einer Wirtschaft möglichst effektiv einzusetzen, um die Wünsche und Bedürfnisse der Menschen bestmöglich zu befriedigen. \\ \\
Wir sprechen von {\bf Produktionseffizienz}, wenn eine Wirtschaft nicht mehr von einem Gut erzeugen kann, ohne zugleich bei einem anderen Abstriche machen zu müssen, wenn sie sich also auf Ihrer PMK (siehe 1.3.3)befindet. {\bf Produktionsineffizienz} bezeichnet den Zustand einer Wirtschaft, wenn sich diese aus unterschiedlichen Gründen nicht auf der PMK befindet.
\\ \\
{\it Erläuterungen: Das Wesen des Wirtschaftens besteht in der Anerkennung der Knappheit als Realität und in einer gesellschaftlichen Organisation, die einen möglichst effizienten Ressourceneinsatz zulässt.}
\subsubsection{Gut \& Wirtschaftsgut}
Als {\bf Gut } im Allgemeinen bezeichnet man in der Wirtschaftswissenschaft alle Mittel, die der Bedürfnisbefriedigung dienen.

Im engeren Sinn versteht man {\bf Güter} als {\bf Wirtschaftsgüter}; diese werden über ihre Knappheit definiert ("knappe Güter"):
\begin{itemize}
\item es handelt sich um ein Gut, das nicht zu jeder Zeit und an jedem gewünschten Ort in der gewünschten Qualität und Menge zur Verfügung steht.
\item Tausch- und Marktfähigkeit.
\end{itemize}

\subsubsection{Mikroökonomie \& Makroökonomie}
{\bf Mikroökonomie} bezeichenet den Teil der Volkswirtschaft, der sich mit dem Verhalten einzelner Wirtschaftseinheiten wie Märkte, Unternehmen oder Haushalten auseinandersetzt.
Das zweite wichtige Teilgebiet der Volkswirtschaftslehre ist die {\bf Makroökonomie}. Im Gegensatz zur Mikroökonomie arbeitet die Makroökonomie mit aggregierten Größen, also zum Beispiel mit dem Gesamteinkommen aller Haushalte. Heute beschäftigt sich die Makroökonomie mit einer breiten Palette an Themen, etwa mit den Faktoren, die die Investitionen und den Konsum einer Gesellschaft bestimmen, oder mit der Frage, welche Kontrolle die Zentralbanken über Geldmenge und Zinssätze ausüben, was zu internationalen Finanzkrisen führt und warum einige Staaten kräftig wachsen, andere aber stagnieren

\subsubsection{Ökonometrie}
Darüber hinaus haben Ökonomen eine
spezielle Technik, die Ökonometrie, entwickelt, die statistische Methoden auf wirtschaftliche Problemstellungen anwendet. Mithilfe der Ökonometrie können die Wissenschaftler aus riesigen Datenbergen einfache Beziehungen ableiten.
Aber Achtung, wie immer wenn was einfach aussieht:
\begin{itemize}
\item  Der „Post-hoc-Irrtum“. Hier wird zu Unrecht eine Kausalität angenommen, wo
keine besteht.\\
{\it Dieser Irrtum tritt auf, wenn wir aufgrund der zeitlichen Aufeinanderfolge zweier Ereignisse annehmen, das zweite sei durch das erste verursacht worden.} 
\item  Der Irrtum, nicht „alles andere konstant“ zubelassen.\\
{\it Ein weiterer verbreiteter Irrtum ist es, bei der Betrachtung eines Faktors nicht alle anderen Faktoren unverändert zu lassen. }
\item  Der Trugschluss der Verallgemeinerung.\\
{\it Manchmal schließen wir von einem Teil voreilig auf das Ganze. }
\end{itemize}

\subsubsection{Positive vs. Normative Ökonomik}
Die positive Ökonomik beschreibt die Fakten einer Wirtschaft, während sich die normative Ökonomik mit Werturteilen befasst.\\
\begin{itemize}
\item {\bf positive Ökonomik}\\
Fragen die sich mit gründlicher Analyse und mithilfe empirischer Daten lösen lassen.
\item {\bf normative Ökonomik}\\
In der normativen Ökonomik geht es um ethisches Verhalten und Normen der Fairness.

\end{itemize}
{\it 
BSP:\\
positiv: Warum verdienen Ärzte mehr als Maurer?\\
normativ: Sollten Bedürftige vom Staat, der sie unterstützt, zum Arbeiten angehalten werden?\\}
\subsection{Die 3 Grundfragen der Wirtschaft}
\begin{itemize}
\item {\bf Was} wird produziert und in welchen Mengen? 
\item {\bf Wie} wird produziert?
\item Für {\bf wen} wird produziert?
\end{itemize}
\subsubsection{ Marktwirtschaft, Planwirtschaft, Mischwirtschaft}
Die drei Grundfragen werden von den Volkswirtschaften verschieden
beantwortet.\\
Man unterscheidet zwischen zwei grundlegend verschiedenen Organisationsformen (Wirtschaftssysteme) einer Wirtschaft:
\begin{itemize}
\item In der Planwirtschaft werden die Grundfragen wie, was und für wen produzieren vom Staat entschieden.
\item In der Marktwirtschaft fallen die Entscheidungen auf den Märkten, wo sich Einzelpersonen (Haushalte) oder Unternehmen freiwillig über den Einsatz von Ressourcen (Input) und das Produktionsergebnis (Output) einigen. Preise, Märkte, Gewinne und Verluste, Anreize und Belohnungen über das Was, Wie und für Wen.
\item Die reinen Formen existieren nicht. Alle Volkswirtschaften verfügen über ein Mischsystem mit Elementen aus Markt- und Planwirtschaft.
\end{itemize}
\subsection{Die Technologischen Möglichkeiten einer Gesellschaft}
\subsubsection{Input, Output}
\begin{itemize}
\item {\bf Inputs} sind Waren oder Dienstleistungen, die ihrerseits der Erzeugung von Gütern und Dienstleistungen dienen. Eine Wirtschaft setzt die ihr zur Verfügung stehenden Technologien ein, um mit Hilfe der Inputs Outputs zu erzeugen.
\item {\bf Outputs } sind die verschiedenen nützlichen Güter und Dienstleistungen, die aus der Produktion hervorgehen und die entweder konsumiert oder im weiteren Produktionsprozess eingesetzt werden.
\end{itemize} 
\subsubsection{Produktionsfaktoren}
\begin{itemize}
\item {\bf natürliche Ressourcen} (Grund und Boden)\\
Zum Produktionsfaktor Boden gehören die Felder, auf denen wir unsere Landwirtschaft betreiben, oder die Grundstücke für unsere Häuser, Fabriken und Straßen; in
diese Kategorie fallen aber auch Energieressourcen für unsere Autos, Heizungen oder Haushalte sowie Bodenschätze wie Kupfer, Erz oder Sand. In einer immer dichter besiedelten Welt müssen wir die Palette natürlicher Ressourcen um Umweltressourcen wie saubere Luft und Trinkwasser erweitern.
\item {\bf Arbeit} \\
Arbeit  als Produktionsfaktor bedeutet die Zeit, die Menschen für die Produktion aufwenden.
\item {\bf Kapital} \\
Kapital beinhaltet jene dauerhaften Güter einer Wirtschaft, die produziert werden, damit andere Güter erzeugt werden können. Zu den Kapital- oder Investitionsgütern gehören Maschinen, Straßen, Computer, Hämmer, LKWs, Stahlwerke, Autos, Waschmaschinen und Gebäude. Wie wir später noch sehen werden, ist es für die wirtschaftliche Entwicklung unabdingbar, einen Grundstock an spezialisierten Investitionsgütern anzusammeln.
\end{itemize}

\subsubsection{Transformations- oder Produktionsmöglichkeitenkurve (PMK) }
\begin{wrapfigure}{l}{0.4\textwidth}
  \begin{center}
    \includegraphics[width=0.38\textwidth]{img/pmk.jpg}
  \end{center}
  \caption{\it Produktionsmöglichkeitenkurve (PMK)}
\end{wrapfigure}
Die Transformations- oder Produktionsmöglichkeitenkurve (PMK) zeigt die maximalen Produktionsmengen, die eine Wirtschaft angesichts ihres technologischen Know-how's und der verfügbaren Menge an Produktionsfaktoren erzielen kann. Die PMK stellt die Gesamtheit der Güter und Dienstleistungen dar, die eine Gesellschaft produzieren kann.\\
{\it Produktionsmöglichkeiten- oder Transformationskurven veranschaulichen eine Vielzahl grundlegender öko-nomischer Prozesse: wie das Wirtschaftswachstum die PMK nach außen verschiebt, wie ein Staat im Zuge seiner Entwicklung vergleichsweise immer weniger Nahrungsmittel und andere lebensnotwendige Güter erzeugt, wie ein Land zwischen privaten und öffentlichen Gütern zu wählen hat und wie sich Gesellschaften zwischen Konsumgütern und Kapitalgütern, die den künftigen Konsum erhöhen, entscheiden müssen.\\
Gesellschaften befinden sich bisweilen innerhalb ihrer PMK. Bei hoher Arbeitslosigkeit oder wenn Unruhen oder ineffiziente staatliche Maßnahmen die Wirtschaftsaktivitäten hemmen, wird die Wirtschaft ineffizient und fällt daher hinter ihre PMK zurück.}\\
\subsubsection{Opporunitätskosten}
In einer Welt der Knappheit bedeutet die Wahl einer Möglichkeit immer den Verzicht auf eine andere. Die {\bf Opportunitätskosten } einer Entscheidung entsprechen dem Wert des nicht gewählten Gutes oder der nicht gewählten Dienstleistung.\\
{\it Tauschware Zeit \\
 Eine der wichtigsten Entscheidungen, die Menschen treffen müssen, betrifft den Einsatz ihrer Zeit. Für sämtliche Aktivitäten, denen Leute nachgehen, steht ihnen immer nur ein beschränktes Angebot an Zeit zur Verfügung. }
 \pagebreak
 
 
\section{Markt und Staat in der modernen Wirtschaft}
\subsection{Markt}
Ein {\bf Markt} ist ein Mechanismus, mit dessen Hilfe Käufer und Verkäufer miteinander in Beziehung treten, um Preis und Menge einer Ware oder Dienstleistung zu ermitteln.
\subsubsection{Preis(e)}
{\bf Preise} koordinieren die Entscheidungen von Produzenten und Konsumenten auf einem Markt. Höhere Preise dämpfen zumeist die Nachfrage bei den Konsumenten und kurbeln gleichzeitig die Produktion an. Geringere Preise hingegen fördern die Kauflust der Leute und wirken sich hemmend auf die Produktion aus. Preise sind also das ausgleichende Element im Marktmechanismus.
\subsubsection{Marktgleichgewicht}
Das Marktgleichgewicht stellt den Ausgleich zwischen all den verschiedenen Käufern und Verkäufern her. Der Markt ermittelt den Gleichgewichtspreis, der sowohl die Wünsche der Käufer als auch jene der Verkäufer berücksichtigt.\\
{\bf Wie lösen Märkte die drei Grundfragen der Volkswirtschaft?}
\begin{itemize}
\item {\bf Was} produziert wird (also welche Güter und Dienstleistungen), entscheiden letztlich die Konsumenten, deren Geldausgaben als eine Art „Stimmzettel“ fungieren.
\item {\bf Wie} produziert wird, entscheidet sich durch den Wettbewerb zwischen verschiedenen Produzenten. Die beste Möglichkeit für die Produzenten, die Preiskonkurrenz zu bestehen und die eigenen Gewinne zu maximieren, besteht in einer Minimierung ihrer Kosten durch den Einsatz möglichst effizienter Produktionsmethoden.
\item {\bf Für wen } produziert wird – also wer konsumiert und wie viel –, hängt weitgehend von Angebot und Nachfrage auf den Faktormärkten ab. Faktormärkte (die Märkte für Produktionsfaktoren) bestimmen die Höhe der Löhne und Pachten, Zinssätze und Gewinne. Die zugehörigen Preise heißen daher {\it Faktorpreise}
\end{itemize}

    \includegraphics[width=0.5\textwidth]{img/markt.jpg}

Wir sehen hier den Kreislauf einer Marktwirtschaft. Die Kaufentscheidungen der Konsumenten (also von Haushalten, öffentlicher Hand und Ausländern) treten in eine Wechselwirkung zum Angebot auf den Gütermärkten (oben) und bestimmen dadurch mit, was produziert wird. Die Unternehmensnachfrage nach Produktionsfaktoren oder Inputs trifft auf den Faktormärkten (unten) auf das Arbeitsangebot und sonstige Inputs, wodurch Löhne, Renten und Zinsen bestimmt werden; auf diese Weise beeinflussen die Einkommen, für wen produziert wird. Der Wettbewerb zwischen den Unternehmen in Bezug auf den möglichst billigen Kauf von Faktorinputs und Verkauf von Gütern entscheidet darüber, wie Güter produziert werden. \\
\subsubsection{Die unsichtbare Hand}
Adam Smith hat eine ganz bemerkenswerte Eigenschaft der wettbewerbsorientierten Marktwirtschaft entdeckt. Unter den Bedingungen des vollständigen Wettbewerbs, und sofern es zu keinem Marktversagen kommt, gelingt es den Märkten, das Maximum an nützlichen Gütern und Dienstleistungen aus den vorhandenen Ressourcen herauszuholen. Wo jedoch Monopole, Umweltverschmutzung oder ähnliche Formen von Marktversagen auftreten, kann dies die bemerkenswerte Effizienz der unsichtbaren Hand untergraben. 

\subsection{Handel, Geld und Kapital}
Reife, entwickelte kapitalistische Volkswirtschaften zeichnen sich durch drei wesentliche Merkmale aus: Handel und Spezialisierung, Geld und Sachkapital.
\begin{itemize}
\item Eine entwickelte Volkswirtschaft verfügt über ein komplexes Handelsnetz zwischen den agierenden Personen und Ländern, das erst mit massiver Spezialisierung und ausgeklügelter Arbeitsteilung möglich wird.
\item Die heutigen, modernen Volkswirtschaften arbeiten in hohem Maße mit Geld oder
anderen Zahlungsmitteln.
\item Die Technologien unserer modernen Industrie beruhen auf dem Einsatz riesiger Mengen an Sachkapital: Präzisionsmaschinen, Großraumfabriken, Lagerbestände
\end{itemize}
\subsubsection{Handel, Spezialisierung und Arbeitsteilung}
Entwickelte Volkswirtschaften bemühen sich um Spezialisierung und Arbeitsteilung, wodurch sie die Produktivität ihrer Ressourcen erhöhen. Einzelpersonen und Länder tauschen freiwillig jene Produkte, auf deren Herstellung sie sich spezialisiert haben, gegen die Produkte anderer und vergrößern auf diese Weise sowohl die Bandbreite als auch die Menge der konsumierten Güter enorm, wodurch in der Folge der Lebensstandard aller steigt.
\subsubsection{Geld}
Geld ist ein Tauschmittel. Der richtige Umgang mit der Geldmenge gehört zu den wichtigsten Aufgaben der Wirtschaftspolitik aller Staaten.
\subsubsection{Kapital}
In der Wirtschaft geht es unter anderem darum, auf sofortigen Konsum zu verzichten und stattdessen das vorhandene Kapital zu vermehren. Jedes Mal, wenn wir etwas investieren – eine neue Fabrik oder Straße bauen, die Ausbildung verlängern oder qualitativ verbessern oder unsere Bestände an nutzbarer Technologie und Know-how erhöhen –, fördern wir die künftige Produktivität unserer Wirtschaft und damit auch den künftigen Konsum.\\
\\
\subsubsection{Kapital und Privateigentum}
In einer Marktwirtschaft befindet sich Kapital zumeist in Privateigentum, und der Einzelne bezieht ein Einkommen aus seinem Kapital. Für jeden Flecken Land gibt es eine Übertragungsurkunde oder einen Eigentumstitel; beinahe jede Maschine und jedes Gebäude gehört einem einzelnen Menschen oder einem Unternehmen. Eigentumsrechte verleihen den Eigentümern die Möglichkeit, ihre Kapitalgüter nach ihrem Ermessen zu verwenden, auszutauschen, anzustreichen, auszugraben, anzubohren oder auszubeuten. Diese Kapitalgüter haben auch einen Marktwert, und man kann sie zu jedem beliebigen erzielbaren Preis auf dem Markt kaufen und verkaufen. Diese Möglichkeit des Einzelnen, Kapital zu besitzen und daraus Nutzen zu ziehen, gibt dem Kapitalismus seinen Namen.

\subsubsection{Moderne Volkswirtschaft}
Eine moderne Volkswirtschaft muss ganz spezielle Eigenschaften aufweisen, um möglichst
produktiv zu sein. Arbeitsteilung und hoch spezialisiertes Kapital erlauben es dem Einzelnen, sich selbst auf einem bestimmten Gebiet zu spezialisieren. Doch all die spezialisierten Individuen, Unternehmen und Länder können nur überleben, weil der mit Geld geschmierte Handel es verschiedenen Personen und Ländern erlaubt, ihre Produkte problemlos zu verkaufen und ebenso problemlos das Nötige einzukaufen. Die Spezialisierung hat eine enorme Effizienz zur Folge; die gesteigerte Produktion ermöglicht den Handel; das Geld macht diesen Handel schnell und effizient; und ein ausgeklügeltes Finanzsystem dient dazu, die Ersparnisse einiger in das Kapital anderer zu verwandeln.

\subsection{Die Rolle des Staates in der Wirtschaft}
Staaten haben in einer Marktwirtschaft drei wesentliche wirtschaftliche Funktionen: Die Erhöhung der Effizienz, die Förderung der sozialen Gerechtigkeit und die Sicherung von volkswirtschaftlicher Stabilität und Wachstum.
\subsubsection{Effizienz}
Eine schwerwiegende Abweichung vom effizienten Markt stellt der {\bf unvollständige Wettbewerb oder das Monopol} dar. Während im vollständigen Wettbewerb weder einzelne Unternehmen noch einzelne Konsumenten Einfluss auf die Preisbildung nehmen können, sprechen wir von unvollständigem Wettbewerb, wenn ein einzelner Käufer oder Verkäufer in der Lage ist, Einfluss auf den Preis einer Ware oder Dienstleistung zu nehmen.\\
{\bf Externe Effekte (oder Spillover-Effekte)} treten auf, wenn die wirtschaftliche Aktivität von Unternehmen oder Individuen bei marktfernen Akteuren zu Kosten oder einem Nutzen führt. (z.B. Flughafenlärm)\\
{\bf öffentliche Güter} Ein klassisches Beispiel dafür ist die Armee eines Landes. Wenn
ein Land in den Krieg zieht – um Terroristen zu bekämpfen, Massenvernichtungswaffen zusuchen, um Land oder Öl zu stehlen oder patriotische Gefühle in der Bevölkerung zu schüren –, müssen alle die Zeche zahlen, ob sie wollen oder nicht. 
\subsubsection{Soziale Gerechtigkeit}
Die Märkte produzieren nicht notwendigerweise eine sozial gerechte Einkommensverteilung. Eine Marktwirtschaft kann sogar völlig inakzeptable Ungleichheiten bei Einkommen und Konsum hervorbringen.
\subsubsection{Wirtschaftswachstum und Stabilität}
\begin{wrapfigure}{l}{0.5\textwidth}
  \vspace{-20pt}
  \begin{center}
    \includegraphics[width=0.48\textwidth]{img/staat.jpg}
  \end{center}
   \vspace{-10pt} 
\end{wrapfigure}
Durch den umsichtigen Einsatz {\bf fiskal- und geldpolitischer Maßnahmen} können die Staaten Produktivität, Beschäftigung und Inflation wirksam beeinflussen. Unter der Fiskalpolitik eines Staates versteht man dessen Möglichkeit, Steuern zu erheben und Ausgaben zu tätigen. Als Geldpolitik bezeichnet man hingegen die Festlegung der Geldmenge und der Zinssätze. \\
Makroökonomische Strategien zur Stabilisierung und Förderung des Wirtschaftswachstums beinhalten fiskal- oder haushaltspolitische Maßnahmen (Steuereinnahmen und Staatsausgaben) ebenso wie geldpolitische Maßnahmen (Einflussnahme auf Zinssätze und Kreditkonditionen). Seit der Entwicklung der Makroökonomie in den 1930er Jahren konnten Staaten immer wieder erfolgreich die schlimmsten Exzesse von Inflation und Arbeitslosigkeit eindämmen.









\pagebreak
\newpage
\section{Angebot und Nachfage}
\subsection{Nachfragefunktion}
Die Beziehung zwischen Preis und gekaufter Menge wird als Nachfragefunktion oder Nachfragekurve bezeichnet.
\subsubsection{Das Gesetz des negativen Nachfrageverlaufs}
Wenn der Preis für eine Ware angehoben wird (und alle anderen Faktoren gleich bleiben), neigen die Käufer dazu, weniger von dieser Ware zu kaufen. Ebenso erhöht sich, wenn der Preis gesenkt wird und alle anderen Einflussfaktoren unverändert bleiben, die nachgefragte Menge.
\subsubsection{Substitutions- / Einkommenseffekt}
\begin{itemize}
\item {\bf Substitutionseffekt }Sobald der Preis einer Ware steigt, werde ich versuchen, diese Ware durch eine andere zu ersetzen.  
\item Der {\bf Einkommenseffekt} kommt ins Spiel, weil ein erhöhter Preis etwas ärmer macht. Der Versuch weniger von einer Ware zu konsumieren beginnt.
\end{itemize}
\subsubsection{Marktnachfrage / -kurve}
Die Marktnachfragekurve lässt sich durch Addition der von allen Konsumenten zu jedem Preis nachgefragten Mengen ermitteln.
\subsubsection{Kräfte hinter der Nachfragekurve}
\begin{itemize}
\item {\bf Durchschnittseinkommen} Mit steigenden Einkommen kaufen die Leute mehr.
\item {\bf Bevölkerungszahl } Eine Zunahme der Bevölkerung treibt die Verkaufszahlen nach oben.
\item {\bf Preise verwandter Güter} Niedrigere Preise erhöhen die Nachfrage auf Gut.
\item {\bf Präferenzen } Trend, Statussymbol.
\item {\bf Spezielle Einflüsse }  Verkehrsaufkommen in Städten, Umwelt, usw.
\end{itemize}

Wenn sich andere Einflussfaktoren auf das Kaufvolumen als der Preis einer Ware ändern, sprechen wir von einer Nachfrageverschiebung. Die Nachfrage steigt (oder sinkt), wenn die zu jedem Preis nachgefragte Menge steigt (oder sinkt). 

{\bf Bewegung entlang einer Kurve oder Verschiebung der Kurve selbst? }\\
Verwechseln Sie bitte nie Bewegungen entlang von Kurven mit Kurvenverschiebungen. Unterscheiden Sie unbedingt zwischen einer Veränderung der Nachfrage (und somit einer Verschiebung der Nachfragekurve) und einer Veränderung der nachgefragten Menge (also der Bewegung hin zu einem  anderen Punkt auf derselben Nachfragekurve infolge einer Preisänderung).  Eine Änderung der Nachfrage ergibt sich, sobald sich einer der Faktoren, die der  Nachfragekurve zugrunde liegen, verändert.
\subsection{Angebotsfunktion}
Die Angebotsfunktion (oder die Angebotskurve) eines Gutes stellt die Beziehung zwischen seinem Marktpreis und jener Menge dieses Gutes dar, die die Produzenten zu produzieren und zu verkaufen gewillt sind,  wenn alle anderen Faktoren konstant bleiben.
\subsubsection{Kräfte hinter der Angebotskurve}
\begin{itemize}
\item {\bf Technologie } Neue Maschienen senken Produktionskosten
\item {\bf Faktorpreise } Niedrigere Löhne der Arbeiter senken die Produktionskosten und erhöhen das Angebot.
\item {\bf Preise verwandter Güter } Sinkende Preise von ähnlichen Gütern sinken
\item {\bf Wirtschaftspolitische Maßnahmen } Die Beseitigung von Importquoten und Zöllen auf importierte Güter.
\item {\bf Spezielle Einflüsse } Einkäufe und Auktionen im Internet ermöglichen den Konsumenten einen einfachen Preisvergleich zwischen den Anbietern und vertreiben teure Anbieter vom Markt.
\end{itemize}
Wenn die Angebotsmenge von anderen Faktoren als dem Preis einer Ware beeinflusst wird, sprechen wir von einer Angebotsverschiebung. Das Angebot steigt (oder sinkt), wenn die zu jedem Marktpreis angebotene Menge steigt (oder sinkt).
\subsection{Marktgleichgewicht und Gleichgewichtspreis}
\begin{wrapfigure}{l}{0.5\textwidth}
  \begin{center}
    \includegraphics[width=0.4\textwidth]{img/angebot_nachfrage_1.jpg}
  \end{center}
\end{wrapfigure}
Ein {\bf Marktgleichgewicht} stellt sich bei dem Preis ein, bei dem die nachgefragte Menge der angebotenen Menge entspricht. Bei diesem Gleichgewicht gibt es keine Preistendenzen nach oben oder unten. Der Gleichgewichtspreis wird auch Markträumungspreis genannt. Damit soll ausgedrückt werden, dass bei diesem Preis alle Angebots- und Nachfragevorstellungen befriedigt werden, dass die Auftragsbestände in den Büchern ausgeglichen sind und dass Konsumenten wie auch Produzenten rundum zufrieden sind.
\\
\\
{\bf Gleichgewichtspreis} und {\bf Gleichgewichtsmenge }stellen sich dort ein, wo die freiwillig angebotene der freiwillig nachgefragten Menge entspricht. Auf einem vollkommenen Markt tritt dieses Gleichgewicht im Punkt der Überschneidung von Angebots- und Nachfragekurve ein. Zum Gleichgewichtspreis gibt es weder einen Überhang noch eine Fehlmenge.
\\
\\
\includegraphics[width=0.75\textwidth]{img/angebot_nachfrage_bewegungen.jpg}
\pagebreak
\section{Anwendungsgebiete Angebots- und Nachfrageanalyse}
\subsection{Preiselastizität von Angebot und Nachfrage}
Damit aus der Theorie von Angebot und Nachfrage ein wirklich brauchbares Werkzeug entsteht, müssen wir wissen, wie stark Angebot und Nachfrage auf Preisänderungen reagieren. Bestimmte Kaufentscheidungen, wie zum Beispiel Urlaubsreisen, sind von Preisänderungen sehr leicht beeinflussbar.Die Elastizität ist tendenziell höher, wenn es sich um Luxusgüter handelt, wenn Ersatzgüter verfügbar sind und wenn die Konsumenten länger Zeit haben, um ihr Verhalten anzupassen. Andere Güter, wie Lebensmittel oder elektrischer Strom, sind Notwendigkeiten des täglichen Lebens, was bedeutet, dass die Kaufbereitschaft der Konsumenten in diesem Sektor kaum auf Preisänderungen reagiert.
\\
Die {\bf Preiselastizität der Nachfrage, Preiselastizität} misst, inwieweit sich die nachgefragte Menge eines Gutes infolge von Preisänderungen verändert.
\subsection{Preiselastizität der Nachfrage}
\begin{equation}
\text{{\bf Preiselastizität der Nachfrage}} = \frac{\text{prozentuale Änderung der nachgefragten Menge}}{\text{prozentuale Preisänderung}} \nonumber
\end{equation}\\
\begin{equation}
E_{D} =  \frac{ \Delta Q }{\frac{(Q_{1} + Q_{2})}{2}} \div \frac{ \Delta P }{\frac{(P_{1} + P_{2})}{2}} \nonumber
\end{equation}
{\it wobei P1 und Q1 den ursprünglichen Preis und die ursprüngliche Menge und P2 und Q2 den neuen Preis und die neue Menge bezeichnen.}\\
\includegraphics[width=0.95\textwidth]{img/elastizitaten.jpg}
\begin{itemize}
\item {\bf preiselastischen Nachfrage:} Wenn eine einprozentige Preisänderung eine mehr als einprozentige Änderung der nachgefragten Menge nach sich zieht, ist die Preiselastizität der
Nachfrage dieses Gutes sehr hoch.
\item {\bf preisunelastischen Nachfrage:} Wenn eine einprozentige Preisänderung eine weniger als einprozentige Änderung der nachgefragten Menge nach sich zieht.
\item Ein wichtiger Sonderfall liegt vor, wenn die Elastizität der Nachfrage den Wert 1 annimmt, wenn also die prozentuale Mengenänderung genau so groß ist wie die prozentuale Preisänderung. Dies impliziert, dass die Gesamtausgaben für das betreffende Wirtschaftsgut trotz Preisänderung gleich bleiben.
\item Alle Punkte auf der gerade verlaufenden Nachfragekurve haben dieselbe Steigung. Hingegen ist die Nachfrage oberhalb von Punkt M elastisch, darunter jedoch unelastisch. Im Mittelpunkt beträgt die Nachfrageelastizität 1.
\end{itemize}
\pagebreak
\subsubsection{Kurzmethode zur Berechnung}
\begin{wrapfigure}{L}{0.18\textwidth}
\vspace{-20pt}
  \begin{center}
    \includegraphics[width=0.16\textwidth]{img/kurzmethode_elas.jpg}
  \end{center}
  \vspace{-20pt}
\end{wrapfigure}
Die Nachfrageelastizität ergibt sich durch das Verhältnis der Länge des geraden Linienabschnitts unterhalb des Punktes zur Länge des Linienabschnitts oberhalb des Punktes. Das bedeutet, dass die Elastizität bei Punkt B mit 3 berechnet werden kann. Für nicht lineare Nachfragekurven zeichnen Sie einfach die Tangente und berechnen Sie ihre Elastizität.\\

\subsection{Elastizität und Ertrag}
Der Gesamtertrag ist definiert als Preis mal Menge (oder P x Q). Kaufen die Konsumenten 5 Einheiten eines Gutes zu je CHF 3, beträgt der Gesamtertrag CHF 15. Wenn Sie die Preiselastizität der Nachfrage kennen, wissen Sie jetzt auch, wie sich der Gesamtertrag nach einer Preisänderung entwickeln wird:
\begin{itemize}
\item Bei preisunelastischer Nachfrage verringert eine Preissenkung den Gesamtertrag.
\item Bei preiselastischer Nachfrage erhöht eine Preissenkung den Gesamtertrag.
\item Im Grenzfall der Nachfrageelastizität von 1 bewirkt eine Preissenkung keinerlei Veränderung des Gesamtertrages
\end{itemize}
\subsection{Preiselastizität der Angebots}
Die Preiselastizität des Angebots misst die prozentuale Änderung des von den Produzenten gelieferten Outputs, wenn sich der Marktpreis um einen bestimmten Prozentsatz ändert.
\begin{equation}
\text{{\bf Preiselastizität des Angebots}} = \frac{\text{prozentuale Änderung der angebotenen Menge}}{\text{prozentuale Preisänderung}} \nonumber
\end{equation}\\
\subsection{Anwendung auf wichtige ökonomische Fragen}
Ein besonders anschauliches Gebiet zum Studium der Wirkungsweise von Angebot und Nachfrage ist die Landwirtschaft. Verbesserungen in der Agrartechnologie bedeuten, dass das Angebot stark steigt, während die Nachfrage nach Lebensmitteln weniger zunimmt, als den Einkommenssteigerungen der Konsumenten entsprechen würde. Daher fallen die Preise für Nahrungsmittel auf dem freien Markt tendenziell. Kein Wunder also, dass viele Staaten eine ganze Reihe von Programmen eingeführt haben, etwa Flächenstilllegungen, um die Einkommen ihrer Landwirte anzuheben.
\\ \\ 
Eine Produktsteuer (Verbrauchssteuer auf ein Wirtschaftsgut) verschiebt das Gleichgewicht von Angebot und Nachfrage. Die Steuerlast (oder die Wirkung auf die Einkommen) trifft die  Konsumenten in dem Ausmaß stärker als die Produzenten, in dem die Nachfrage im Vergleich zum Angebot unelastisch ist.
\\ \\
{\bf Izidenz: }  Mit Inzidenz meinen wir die letztendliche Auswirkung einer Steuer auf die Realeinkommen der Produzenten oder Konsumenten. 
\\ \\
Staaten greifen von Zeit zu Zeit in die Mechanismen der Wettbewerbsmärkte ein, indem sie Höchst- oder Mindestgrenzen für Preise festsetzen. In einer solchen Situation muss die angebotene Menge nicht mehr  der nachgefragten Menge entsprechen; Preisobergrenzen führen zu einem Nachfrageüberschuss, während Preisuntergrenzen einen Angebotsüberschuss zur Folge haben. Manchmal kann der Eingriff die Einkommen einer bestimmten Gruppe erhöhen,  wie es bei Landwirten oder bei Arbeitskräften in Niedriglohnbranchen der Fall ist. Verzerrungen und Ineffizienz sind oft die Folge.
\section{Nachfrage und Konsumverhalten}
\subsection{Nutzen}
{\bf Nutzen} bedeuted Berdürfnissbefriedigung. Genauer gesagt, beschreibt der Begriff,wie die Konsumenten verschiedene Güter und Dienstleistungen bewerten. Der Begriff des Nutzens ist ein wissenschaftliches Konstrukt, das es den Ökonomen ermöglicht zu verstehen, wie die Konsumenten ihre beschränkten Ressourcen auf die Waren verteilen, die ihre Bedürfnisse befriedigen.\\
Das {\bf Gesetz des abnehmenden Grenznutzens} besagt, dass der Grenznutzen mit zunehmender Menge eines konsumierten Gutes in aller Regel abnimmt.\\ 
\\
Der Begriff „Grenz-“ ist in der „Volkswirtschaftslehre von größter Bedeutung, und er wird immer im Sinne von „zusätzlich“ verwendet. Der Grenznutzen beschreibt den zusätzlichen Nutzen, den Ihnen der Konsum einer zusätzlichen Einheit eines Gutes verschafft. \\
\begin{wrapfigure}{l}{0.5\textwidth}
\vspace{-20pt}
  \begin{center}
    \includegraphics[width=0.48\textwidth]{img/nutzen.jpg}
  \end{center}
  \vspace{-10pt}
\end{wrapfigure}
Der Gesamtnutzen in (a) steigt mit dem Konsum, aber er steigt nicht genauso stark wie der Konsum, was den abnehmenden Grenznutzen signalisiert. Diese Beobachtung hat die Ökonomen früherer Zeiten dazu bewogen, das Gesetz des negativen Nachfrageverlaufs zu formulieren. Die grauen Blöcke zeigen den durch jede neu hinzukommende Einheit bewirkten Zusatznutzen. Die Tatsache, dass der Gesamtnutzen in immer geringerem Maß steigt, wird in (b) durch die abwärts verlaufenden Stufen des Grenznutzens dargestellt. Wenn wir unsere Einheiten immer kleiner machen, werden die Stufen letztlich geglättet, sodass der Gesamtnutzen zu der  durchgehenden schwarzen Kurve in (a) wird. Außerdem lässt sich der durchgehende Grenznutzen, der in (b) durch die schwarze abfallende Kurve dargestellt wird, von der Steigung der durchgehenden Kurve in (a) nicht unterscheiden. \\
\subsection{Prinzip des gleichen Grenznutzens}
Die grundlegende Bedingung für die größtmögliche Bedürfnisbefriedigung oder den größtmöglichen Nutzen ist {\bf das Prinzip des gleichen Grenznutzens}. Laut diesem Prinzip erzielt ein Konsument mit einem gegebenen Einkommen, der mit gegebenen Marktpreisen konfrontiert ist, die maximale Bedürfnisbefriedigung oder den maximalen Nutzen, wenn der Grenznutzen der letzten für jedes Gut ausgegebenen Geldeinheit genau derselbe ist wie der Grenznutzen der letzten für ein anderes Gut ausgegebenen Geldeinheit.\\
\begin{equation}
\text{MU (Grenznutzen) pro Geldeinheit/Einkommen} = \frac{MU_{Gut1}}{P_{1}} = \frac{MU_{Gut2}}{P_{2}} = \frac{MU_{Gut3}}{P_{3}} \nonumber
\end{equation}\\
Der durchschnittliche Grenznutzen pro Geldeinheit aller Güter im Konsumgleichgewicht wird als {\bf Einkommens-Grenznutzen} bezeichnet.\\
Ein höherer Preis für ein Gut reduziert den vom Konsumenten gewünschten Konsum dieses Gutes; dies zeigt, warum die Nachfragekurve negativ verläuft und daher fällt.

\subsection{Alternative: Substitutions- und Einkommenseffekt}
\subsubsection{Substitutionseffekt}
Der {\bf Substitutionseffekt} besagt, dass der Konsument bei steigendem Preis eines Gutes
dazu tendiert, dieses teurere Gut durch andere Güter zu ersetzen, um seine Bedürfnisse auf billigere Weise zu befriedigen.
\subsubsection{Einkommenseffekt}
Der {\bf Einkommenseffekt} bezeichnet die Auswirkung einer Preisänderung auf die nachgefragte Menge eines Gutes, die infolge der Veränderung der realen Einkommen der Konsumenten eintritt. \\
Um ein quantitatives Maß für den Einkommenseffekt zu erhalten, untersuchen wir die
{\bf Einkommenselastizität } eines Gutes. Einkommenselastizität bedeutet die relative, also prozentuale Änderung der nachgefragten Menge dividiert durch die prozentuale Änderung des Einkommens, wobei alle anderen Faktoren, beispielsweise die Preise, konstant bleiben.\\

\subsubsection{Gesamtnachfrage}
\begin{wrapfigure}{l}{0.5\textwidth}
\vspace{-20pt}
  \begin{center}
    \includegraphics[width=0.48\textwidth]{img/kombinierte-nachfrage.jpg}
  \end{center}
  \vspace{-20pt}
\end{wrapfigure}
Die Marktnachfragekurve ist die Summe der Einzelnachfragen zu jedem Preis. Abbildung 5-2 zeigt, wie die einzelnen Nachfragekurven dd horizontal zu addieren sind, um die 
Marktnachfragekurve DD zu erhalten. \\ \\ \\ \\ \\
\subsubsection{Güterklassifizerung}
\begin{itemize}
\item Güter sind {\bf Substitute}, wenn eine Preiserhöhung bei einem Gut die Nachfrage nach dem anderen verstärkt.
\item Güter sind {\bf Ergänzungsprodukte}, wenn eine Preiserhöhung bei einem Gut die Nachfrage nach dem anderen senkt.
\item Güter sind {\bf unabhängig}, wenn eine Preisänderung bei einem Gut keine Auswirkungen auf die Nachfrage nach dem anderen Gut hat. 
\end{itemize}
\subsubsection{Beispiel: Suchtmittel Nachfrage}
\includegraphics[width=0.6\textwidth]{img/junkies.jpg}
\newpage
\subsection{Konsumrente}
\begin{wrapfigure}{l}{0.3\textwidth}
\vspace{-20pt}
  \begin{center}
    \includegraphics[width=0.28\textwidth]{img/konsumrente.jpg}
  \end{center}
  \vspace{-20pt}
\end{wrapfigure}
Da die Konsumenten für alle konsumierten Einheiten nur den Preis der letzten Einheit bezahlen, erzielen sie einen Nutzengewinn gegenüber den Kosten. Die Konsumentenrente entspricht dem zusätzlichen Wert, den die Konsumenten gegenüber dem Preis erzielen, den sie für ein Wirtschaftsgut bezahlt haben.\\
{\it Bild: Die Nachfragekurve misst, wie viel die Konsumenten freiwillig für jede konsumierte Einheit bezahlen würden. Daher zeigt der gesamte Bereich unter der Nachfragekurve (0REM) den Gesamtnutzen aus dem Konsum von Wasser. Indem man die Marktkosten des Wassers für die Konsumenten (entsprechend 0NEM) subtrahiert, erhält man die Konsumentenrente aus dem Wasserkonsum in Form des grauen Dreiecks NER. Diese Methode ist für die Messung des Nutzens öffentlicher Güter und der Verluste aus Monopolen und Importzöllen überaus nützlich.}


\section{Produktion und ihre Organisation im Unternehmen}
\subsection{Grundlagen}
\subsubsection{Produktionsfunktion}
Die Produktionsfunktion sagt aus, welche maximale Produktionsmenge bei gegebenem Faktoreinsatz erzeugt werden kann. Sie gilt jeweils für einen bestimmten Stand der Technik und des technologischen Know-hows. 
\subsubsection{Gesamt-, Durchschnitts- und Grenzprodukt}
\includegraphics[width=0.48\textwidth]{img/grenzprodukt.jpg}
{\bf Gesamtprodukt}, die Gesamtanzahl der produzierten Outputs in physischen Einheiten (Tuben bei Zahnpasta). Das {\bf Grenzprodukt } oder die Grenzproduktivität eines Inputs oder Produktionsfaktors entspricht der zusätzlichen Menge oder dem zusätzlichen Output, die bzw. der durch eine zusätzliche Einheit erzeugt werden kann, während alle anderen Faktoren konstant bleiben. {\bf Durchschnittsprodukt}, der Gesamtoutput dividiert durch die gesamten Inputeinheiten.
\subsubsection{Das Gesetz der abnehmenden Grenzerträge (Ertragsgesetz)}
Das Ertragsgesetz besagt, dass wir laufend geringere zusätzliche Erträge erhalten, wenn wir einen Input bei unveränderten sonstigen Faktoren immer weiter erhöhen. Mit anderen Worten, das Grenzprodukt jeder Inputeinheit sinkt, wenn sich die Menge dieses Inputs erhöht, während alle anderen Faktor konstant bleiben.\\
\\
{\it Das Ertragsgesetz ist eher eine weithin beobachtete empirische Gesetzmäßigkeit als eine universelle Wahrheit, wie wir sie etwa dem Gesetz der Schwerkraft zubilligen.}
\subsection{Skalenerträge}
Die Produktion zeigt konstante, abnehmende oder zunehmende Skalenerträge, je nachdem ob eine gleichmäßige Erhöhung aller Faktoren zu einem überproportionalen, unterproportionalen oder proportionalen Anstieg der Produktionsmenge führt.\begin{itemize}
\item {\bf Konstante Skalenerträge} liegen vor, wenn eine Änderung aller Inputs zu einer Veränderung des Outputs im selben Verhältnis führt. 
\item {\bf Zunehmende Skalenerträge} treten ein, wenn eine Erhöhung aller Inputs zu einem überproportionalen Anstieg des Outputs führt. 
\item {\bf Abnehmende Skalenerträge } treten auf, wenn eine gleichmäßige Erhöhung aller Inputs zu einer unterproportionalen Erhöhung des Gesamtoutputs führt.
 \end{itemize}

\subsection{Kurzfristige und Langfristige Betrachtungsweise}
Für eine effiziente Produktion sind sowohl Zeit als auch konventionelle Produktionsfaktoren wie Arbeit erforderlich. Wir unterscheiden daher in Produktion und Kostenanalyse zwischen zwei unterschiedlich langen Zeitperioden. Als {\bf kurzfristig} wird dabei ein Zeitraum betrachtet, innerhalb dessen nur einige, nämlich die {\it variablen Faktoreinsätze} angepasst werden können. Fixe Faktoren wie Anlagen und Ausstattung lassen sich dagegen kurzfristig nicht vollständig ändern oder anpassen. Als  {\bf langfristig} wird ein Zeitraum bezeichnet, in dem alle fixen und variablen Faktoren, derer sich das Unternehmen bedient, einschließlich Kapital, verändert werden können.

\subsection{Technologischer Wandel}
Wir unterscheiden zwischen {\bf Prozessinnovation} , die dann gegeben ist, wenn neues technisches Know-how die Produktionstechnik für bestehende Produkte verbessert, und {\bf Produktinnovation}, wenn neue oder verbesserte Produkte auf den Markt gebracht werden.  

\subsection{Produktivität und aggregierte Produktionsfunktion}
\subsubsection{Produktivität}
{\bf Produktivität} ist ein Konzept zur Messung des Verhältnisses zwischen dem Gesamtoutput und einem gewichteten Inputdurchschnitt. Zwei bedeutende Varianten der Produktivität sind die {\bf Arbeitsproduktivität}, die den Output je Arbeitseinheit errechnet und die {\bf Gesamtfaktorproduktivität}, die den Output je Einheit der Gesamtinputs misst (typischerweise bestehend aus Kapital und Arbeit).
\subsubsection{Produktivitätswachstum durch Skaleneffekte}
Die Produktivität wächst aufgrund von Skaleneffekten und technologischem Fortschritt. Skaleneffekte und Massenproduktion waren im vergangenen Jahrhundert bedeutende Produktivitätsfaktoren. Die meisten Produktionsprozesse sind heute sehr viel umfangreicher als im 19. Jahrhundert. Ein großes Schiff konnte um die Mitte des 19. Jahrhunderts 2.000 Tonnen Güter befördern, während die heutigen Supertanker mehr als 1 Million Tonnen Öl laden und befördern. 
\subsection{Unternehmen}
Unternehmen sind spezialisierte Gebilde, die sich mit dem Management des Produktionsprozesses befassen. Die Produktion wird deshalb in Unternehmen organisiert, weil Effizienz im Regelfall Produktion im großen Maßstab, die Beschaffung erheblicherfinanzieller Mittel und die Überwachung des laufenden Betriebs erfordert.
\subsubsection{Einzelfirma}
Am unteren Ende des Spektrums befindet sich die Einzelfirma, der klassische Kleinbetrieb, den wir alle kennen. In einem kleinen Laden werden vielleicht ein paar hundert Dollar täglich umgesetzt, und er ernährt seinen Besitzer nur mit Mühe.
\subsubsection{Personengesellschaft}
Ein Unternehmen verlangt häufig nach einer entsprechenden Kombination von Fähigkeiten – etwa nach Rechtsanwälten oder Ärzten, die sich auf unterschiedlichen Gebieten spezialisiert haben. Tun sich zwei oder mehr Personen zusammen, können sie eine Personengesellschaft gründen. diese Unternehmensform bringt bestimmte Nachteile mit sich, die sie für Großbetriebe ungeeignet macht. Ihr Hauptnachteil ist die unbeschränkte Haftung.
\subsubsection{Kapitalgesellschaft}
\begin{itemize}
\item Das Eigentum an der Kapitalgesellschaft ergibt sich aus dem Eigentum an den Stammaktien der Gesellschaft.
\item Im Prinzip kontrollieren die Aktionäre die Kapitalgesellschaften, die sich in ihrem Eigentum befinden.
\item Die Vorstände und Aufsichtsräte der Kapitalgesellschaften sind rechtlich befugt, Entscheidungen für die Gesellschaft zu treffen.
\end{itemize}
Eine effiziente Produktion erfordert häufig große Unternehmenseinheiten, die Kapitalinvestitionen in Milliardenhöhe benötigen. Kapitalgesellschaften mit ihrer beschränkten Haftung und praktischen Managementstruktur können sehr viel privates Kapital anziehen, eine ganze Palette verwandter Güter produzieren und auch das Anlegerrisiko auf viele Schultern verteilen. 

\section{Kostenanalyse}
\subsection{Gesamtkosten: Fixkosten und variable Kosten}
{\bf Gesamtkosten } stellen die in Geld ausgedrückten Mindestgesamtausgaben dar, die benötigt werden, um eine bestimmte Produktionsmengeq zu erzielen. Die {\it Gesamtkosten TC}  steigen, sobald q steigt. \\
{\bf Fixkosten } stellen den gesamten, in Geld ausgedrückten Ausgabenbetrag dar, der selbst dann aufgewendet werden muss, wenn keine Produktionsleistung erzielt wird. Fixkosten bleiben durch Änderungen der Produktionsmenge unbeeinflusst.\\
{\bf Variable Kosten } sind Ausgaben, die mit der Produktionsmenge variieren. Dazu gehören Rohmaterialien, Löhne und Treibstoffe, eben alle Kosten, die keine Fixkosten sind. \\
Per definitionem gilt daher immer:
\begin{equation}
TC = FC + VC \nonumber
\end{equation}
\subsection{Grenzkosten}
\begin{wrapfigure}{l}{0.3\textwidth}
\vspace{-20pt}
  \begin{center}
    \includegraphics[width=0.28\textwidth]{img/grenzkosten_einfach.jpg}
  \end{center}
  \vspace{-20pt}
\end{wrapfigure}
Die Grenzkosten sind eines der zentralen volkswirtschaftlichen Konzepte. Mit diesem Begriff, Grenzkosten (MC), werden alle zusätzlichen Kosten bezeichnet, die bei der Erzeugung jeweils einer zusätzlichen Produktionseinheit anfallen. Produziert ein Unternehmen beispielsweise 1.000 CDs zu Gesamtkosten von US-\$ 10.000 und kostet es US-\$ 10.006, eine mehr, nämlich 1.001 CDs herzustellen, so liegen die Grenzkosten für die Produktion der 1.001. CD bei US-\$ 6.\\ \\
\subsubsection{Grenzkosten im Diagramm}
\begin{wrapfigure}{l}{0.5\textwidth}
\vspace{-20pt}
  \begin{center}
    \includegraphics[width=0.48\textwidth]{img/grenzkosten_diagramm.jpg}
  \end{center}
  \vspace{-10pt}
\end{wrapfigure}
In der vorliegenden Abbildung sind die Daten aus Tabelle 7-2 im Diagramm dargestellt. Die Grenzkosten in (b) werden berechnet, indem man die Zusatzkosten, die laut (a) für jede zusätzliche Produktionseinheit anfallen, feststellt. Um daher die Produktions-MCfür die fünfte Einheit ermitteln zu können, subtrahieren wir US-\$ 160 von US-\$ 210 und erhalten MC = US-\$ 50. Eine durchgängige schwarze Kurve wurde in (a) durch die einzelnen TC-Punkte gezogen, und eine durchgängige schwarze Kurve verbindet auch in (b) die unterschiedlichen MC-Stufen.\\ \\
\includegraphics[width=0.68\textwidth]{img/gesamtkosten.jpg}
\begin{itemize}
\item Die Gesamtkosten setzen sich aus fixen und variablen Kosten zusammen.
\item Die rote Grenzkostenkurve fällt erst ab und steigt dann wieder an, wie durch die MC-Daten in Spalte (5) der Tabelle weiter unten angegeben. Bitte beachten Sie, dass MC die AC-Kurve in ihrem Minimum schneidet.
\end{itemize}

\subsection{Durchschnittskosten}
{\bf Durchschnittskosten} sind die Gesamtkosten, dividiert durch die Gesamtanzahl der produzierten Einheiten. Daher gilt:\\
\begin{equation}
\text{Durchschnittskosten} = \frac{\text{Gesamtkosten}}{\text{Produktionsmenge}} = \frac{TC}{q} = AC \nonumber
\end{equation}
\subsection{Weitere Kostendefinitionen}
\begin{itemize}
\item {\bf Durschnittsfixkosten}
\begin{equation}
AFC = \frac{FC}{q} \nonumber
\end{equation}
\item {\bf durchschnittliche variable Kosten}
\begin{equation}
AVC = \frac{VC}{q} \nonumber
\end{equation}
\end{itemize}
\newpage
\subsubsection{Tabelle}
\begin{wrapfigure}{l}{0.5\textwidth}
\vspace{-20pt}
  \begin{center}
    \includegraphics[width=0.48\textwidth]{img/kosten.jpg}
  \end{center}
  \vspace{-20pt}
\end{wrapfigure}
Wenn die Grenzkosten unter den Durchschnittskosten liegen, senken Sie die Durchschnittskosten.\\
Wenn die MC über den AC liegen, erhöhen Sie die AC.\\
Wenn die MC genau den AC entsprechen, fallen oder steigen die AC in diesem Punktnicht und befinden sich auf ihrem Mindestniveau. Daher gilt am tiefsten Wert einer U-förmigen AC-Kurve: MC = AC = geringstmögliche AC.\\ \\ \\ \\  \\ \\ 
\subsection{Abnehmende Erträge und U-förmige Kostenkurven}
\begin{itemize}
\item
{\bf Kurzfristig } bedeutet einen Zeitraum, der lang genug ist, um die variablen Produktionsfaktoren wie Material und Arbeit anzupassen, jedoch zu kurz, um alle Produktionsfaktoren variieren zu können. Kurzfristig können fixe oder Overhead-Faktoren wie Anlagen und Betriebsausstattung nicht vollständig verändert oder angepasst werden. Deshalb sind die Arbeits- und Materialkosten im Normalfall variable Kosten, während die Kapitalkosten zu den Fixkosten gehören.
\item {\bf Langfristig} lassen sich alle Inputs anpassen – Arbeit ebenso wie Rohstoffe und Kapital. Langfristig sind daher alle Kosten variabel und nicht fix. 
\end{itemize}
Kurzfristig, wenn bestimmte Faktoren wie das Kapital fix sind, weisen die variablen Faktoren eine erste Phase steigender Grenzerträge auf, worauf eine Phase abnehmender Erträge folgt. Die entsprechenden Kostenkurven zeigen eine erste Phase abnehmender Grenzkosten, worauf eine Phase steigender Grenzkosten folgt, sobald die Grenzerträge sinken.\\ \\
\includegraphics[width=0.88\textwidth]{img/ertrage.jpg}
\subsection{Least-Cost-Regel}
Um eine bestimmte Produktionsleistung mit den geringstmöglichen Kosten erzielen zu können, muss ein Unternehmen bei der Beschaffung der Produktionsfaktoren darauf achten, dass das Grenzprodukt pro Geldeinheit, die für jeden einzelnen Produktionsfaktor ausgegeben wird, gleich hoch ist. Das heißt: (siehe Abb. 7-4 für A,L)\\
\begin{equation}
\frac{\text{Grenzpunkt von L}}{\text{Preis von L}} = \frac{\text{Grenzpunkt von A}}{\text{Preis von A}}  \nonumber
\end{equation}
\subsection{Substitutionsregel}
Wenn der Preis für einen Produktionsfaktor sinkt, während alle anderen Faktorpreise gleich bleiben, profitiert ein Unternehmen davon, den verbilligten Faktor anstelle der sonstigen Optionen einzusetzen.


\subsection{Volkswirtschaftliche und betriebswirtschaftliche Kostenrechnung}
\begin{equation}
\textbf{Nettoertrag} = \text{Gesamterlös - Gesamtaufwand} \nonumber
\end{equation}
\includegraphics[width=0.5\textwidth]{img/kostenrechnung.jpg}
\includegraphics[width=0.5\textwidth]{img/bilanz.jpg}
{\bf Unterschied Bilanz und Gewinn/Verlustrechnung}\\
Die Gewinn- und Verlustrechnung misst die Ströme {\it Flussgrösse} in ein Unternehmen hinein und aus ihm heraus, während die Bilanz den Bestand {\it Bestandsgrösse}  an Aktiva und Passiva zum Ende eines Geschäftsjahres darstellt.
\begin{itemize}
\item[1] Verkaufserlöse {Punkt 1}
\item[2] Aufwendungen {Punkte 2-8}
\item[3] Nettoerträge {Punkte 15}
\end{itemize}
\subsubsection{Begriffserläuterungen}
Die {\bf Abschreibung} ist ein Maß für die jährlichen Kosten eines Kapital-Inputs, den ein Unternehmen selbst besitzt. Dieselbe Überlegung gilt für sämtliche Kapitalgüter eines Unternehmens.\\
Formeln zur Berechnung der jährlichen Abschreibung, aber sie alle folgen zwei wichtigen Prinzipien:\\ (a) Die Abschreibung des betreffenden Wertes muss den historischen Kosten oder dem Anschaffungspreis entsprechen;\\ (b) die Abschreibung wird während der buchhalterischen Nutzungsdauer, die sich an der tatsächlichen Nutzungsdauer orientiert, in Form jährlicher Aufwendungen verbucht. \\ 
\\
So gehört zum betrieblichen Rechnungswesen auch die {\bf Bilanz }, die ein Bild von der jeweiligen finanziellen Situation eines Unternehmens zu einem bestimmten Zeitpunkt zeichnet.\\
Auf der einen Seite der Bilanz stehen die {\bf Aktiva} (Vermögenswerte und Rechte des Unternehmens). Auf der anderen, der Passivseite, finden sich zwei Positionen, nämlich die {\bf Verbindlichkeiten }(Schulden und sonstige Verpflichtungen des Unternehmens) und das {\bf Eigenkapital }(der Nettowert, der den Gesamtvermögenswerten abzüglich der gesamten Verbindlichkeiten entspricht).\\
Die wichtigste Annahme, die jeder Bilanz zugrunde gelegt wird, lautet, dass der Wert beinahe jeder Position deren Anschaffungskosten (historischen Kosten) entspricht. \\
\begin{equation}
\text{Gesamtvermögenswerte (Aktiva)} = \text{Gesamtverbindlichkeiten + Eigenkapital} \nonumber
\end{equation}
\begin{equation}
\text{Eigentkapital} = \text{Vermögen (Aktiva) - Verbindlichkeiten} \nonumber
\end{equation}
\subsection{Opportunitätskosten}
Beim Treffen von Entscheidungen fallen Opportunitätskosten an, weil die Auswahl einer Möglichkeit in einer Welt der Knappheit bedeutet, dass wir auf andere Möglichkeiten verzichten müssen. Opportunitätskosten bezeichnen den Wert des wertvollsten entgangenen Gutes oder der entgangenen Dienstleistung.\\
Die volkswirtschaftlichen Kosten berücksichtigen zusätzlich zu den tatsächlichen Geldausgaben alle Opportunitätskosten, die entstehen, weil Ressourcen auch anderweitig eingesetzt werden könnten. \\
{\it Auf gut funktionierenden Märkten entsprechen die Preise den Opportunitätskosten. }
\section{Analyse des Marktes bei vollkommenem Wettbewerb, also nie...}
\subsection{Angebotsverhalten von Unternehmen}
\begin{wrapfigure}{l}{0.3\textwidth}
  \begin{center}
    \includegraphics[width=0.28\textwidth]{img/nachfrage_voll_wett.jpg}
  \end{center}
\end{wrapfigure}
Der vollständige Wettbewerb ist eine Welt der Preisnehmer. Ein im vollständigen Wettbewerb stehendes Unternehmen verkauft ein homogenes Produkt (eines, das mit dem von anderen Branchenteilnehmern verkauften Produkt identisch ist). Das Unternehmen ist im Vergleich zu seinem Markt so klein, dass es keinen Einfluss auf den Marktpreis nehmen kann; es betrachtet den Preis daher als eine gegebene Größe.\\
Da ein Wirtschaftszweig unter Wettbewerbsbedingungen von Unternehmen bevölkert wird, die gemessen am Gesamtmarkt klein sind, entspricht die Nachfragekurve dieses Unternehmens nur einem winzigen Segment der Nachfragekurve der Gesamtbranche. Grafisch dargestellt ist der Anteil des einzelnen Unternehmens an der Nachfragekurve so gering, dass aus der Froschperspektive des Marktteilnehmers im vollständigen Wettbewerb die Nachfragekurve dd absolut horizontal und unbegrenzt elastisch erscheint.\\
\begin{itemize}
\item Unter den Bedingungen des vollständigen Wettbewerbs gibt es zahlreiche kleine Unternehmen, die ein identisches Produkt erzeugen und jeweils zu klein sind, um auf den Marktpreis Einfluss zu nehmen. 
\item Für das Unternehmen im vollständigen Wettbewerb gilt eine völlig waagrechte Nachfragekurve.
\item Der zusätzliche Erlös aus jeder zusätzlich verkauften Einheit entspricht daher dem Marktpreis.
\end{itemize} 
\subsubsection{Angebot unter Wettbewerbsbedingungen}
\begin{wrapfigure}{l}{0.5\textwidth}
  \begin{center}
   \vspace{-20pt}
    \includegraphics[width=0.48\textwidth]{img/hans.jpg}
      \vspace{-20pt}
  \end{center}
\end{wrapfigure}
Der maximale Gewinn ergibt sich bei jener Produktionsmenge, bei der die Grenzkosten dem Preis entsprechen. \\
Regel für die Angebotsmenge eines Unternehmens unter den Bedingungen vollständigen Wettbewerbs: Ein Unternehmen, das nach Gewinnmaximierung strebt, wählt eine Produktionsmenge, bei der die Grenzkosten dem Preis entsprechen:\\
Grenzkosten = Preis oder MC = P\\
Im Allgemeinen kann ein Unternehmen mithilfe seiner Grenzkostenkurve seine optimale Produktionsmenge ermitteln: Die Gewinnmaximierung wird bei jener Produktionsmenge erreicht, bei der Preis- und Grenzkostenkurve einander schneiden.
\subsubsection{Break-Even-Punkt}
Beschreibt jene Produktionsmenge, bei der das Unternehmen einen Gewinn von Null erzielt. Im Break-even-Punkt entspricht der Preis den Durchschnittskosten, daher decken die Erlö- se gerade die Kosten ab.  

\subsubsection{Gewinnmaximierung - allgemein}
Ein Unternehmen, das auf Gewinnmaximierung bedacht ist, entscheidet sich für jene Produktionsmenge, bei der die Grenzkosten dem Preis entsprechen. Im Diagramm bedeutet das, dass die Grenzkostenkurve eines Unternehmens seiner Angebotskurve entspricht.\\
\begin{wrapfigure}{r}{0.3\textwidth}
  \begin{center}
    \includegraphics[width=0.28\textwidth]{img/betriebsminimum.jpg}
    \vspace{-20pt}
    \end{center}
\end{wrapfigure}
\subsubsection{Betriebsminimum}
Die kritische Marktpreisuntergrenze, bei der die Erlöse genau den variablen Kosten entsprechen (oder bei der die Verluste genau den Fixkosten entsprechen), wird als Betriebsminimum bezeichnet.
\subsubsection{Betriebseinstellungsregel}
 Das Betriebsminimum ist der Punkt, an dem die Erlöse die variablen Kosten gerade abdecken oder an dem die Verluste den Fixkosten entsprechen. Wenn der Preis so weit fällt, dass der Preis geringer ist als die durchschnittlichen variablen Kosten, kann das Unternehmen durch Betriebseinstellung seine Gewinne maximieren (seine Verluste minimieren).\\
 \\
Graphik: {\it Die Angebotskurve des Unternehmens entspricht seiner MC-Kurve, solange der Erlös höher ist als die variablen Kosten. Sobald der Preis unter das Betriebsminimum Ps fällt, übersteigen die Verluste die Fixkosten, und das Unternehmen stellt die Produktion ein. Daher stellt die durchgängige rostfarbene Kurve die Angebotskurve des Unternehmens dar.}

\subsection{Das Angebotsverhalten ganzer Wirtschaftszweige}
Um die Marktangebotskurve für ein Gut zu ermitteln, müssen wir die Angebotskurven aller einzelnen Produzenten dieses Gutes horizontal addieren.Die horizontale Addition der Produktionsmengen beim jeweiligen Preis ergibt die Branchen-Angebotskurve.

\begin{wrapfigure}{r}{0.4\textwidth}
  \begin{center}
  	\vspace{-20pt}
    \includegraphics[width=0.38\textwidth]{img/marshall.jpg}
    \vspace{-20pt}
    \end{center}
\end{wrapfigure}
Wir unterscheiden zwischen (a) Perioden, in denen die Unternehmen nur für Anpassungen beim Einsatz des Faktors Arbeit und der anderen variablen Faktoren genügend Zeit haben (kurzfristiges Gleichgewicht) und (b) Perioden, in denen die vollständige Anpassung des Einsatzes der Faktoren – und zwar der fixen wie der variablen – möglich ist (langfristiges Gleichgewicht). Je mehr Zeit für Anpassungen zur Verfügung steht, desto  größer fällt die Elastizität der Angebotsreaktion und desto geringer die Preissteigerung aus. 

\subsubsection{Addition von Angebotskurven}
\includegraphics[width=0.98\textwidth]{img/addition.jpg}
\subsubsection{Langfristiges Break-even-Gleichgewicht}
Bei einer im Wettbewerb stehenden Branche mit identischen, frei in den Markt ein- und austretenden Unternehmen lautet die langfristige Gleichgewichtsbedingung wie folgt: Für jedes identische Unternehmen entspricht der Preis den Grenzkosten sowie den langfristigen Mindest-Durchschnittskosten: \\
P = MC = minimale langfristige AC = Breakeven-Preis\\
Dies ist die Bedingung, unter der der Gewinn langfristig gegen null tendiert („zero-economic-profit“).
Was die langfristige Rentabilität des wettbewerbsorientierten Kapitalismus anbelangt,
kommen wir zu einem überraschenden Schluss. Wir stellen fest, dass die Wettbewerbskräfte die Unternehmen langfristig in Richtung Gewinnschwelle treiben. Auf lange Sicht erzielen im Wettbewerb stehende Unternehmen eine normale Rendite auf ihre Investitionen, aber nicht mehr. Gewinnträchtige Wirtschaftszweige ziehen neue Unternehmen an, was zu Preissenkungen und niedrigeren Gewinnen führt, bis sich die Gewinne gegen null bewegen. Im Gegensatz dazu streben Unternehmen in nicht gewinnträchtigen Branchen nach besseren Renditechancen in anderen Wirtschaftszweigen; Preise und Gewinne tendieren daraufhin nach oben. Im langfristigen Gleichgewicht einer im Wettbewerb stehenden Branche wird daher kein Gewinn erzielt.
\subsection{Allgemeine Regeln}
\subsubsection{Nachfrageregel}
In aller Regel treibt eine Erhöhung der Nachfrage nach einem Gut bei unveränderter Angebotskurve den Preis dieses Gutes in die Höhe. Bei den meisten Gütern bewirkt eine erhöhte Nachfrage auch eine Erhöhung der nachgefragten Menge. Ein Rückgang der Nachfrage hat die gegenteilige Wirkung.
\subsubsection{Angebotsregel}
Ein vermehrtes Angebot eines Gutes (bei konstanter Nachfragekurve) führt im Allgemeinen zu einer Preissenkung und einer Erhöhung der Kauf- und Verkaufsmenge. Ein Angebotsrückgang hat den gegenteiligen Effekt.\\
\includegraphics[width=0.88\textwidth]{img/kostenarten.jpg}
\subsubsection{Angebotsverschiebungen}
\begin{itemize}
\item  Ein vermehrtes Angebot senkt P dann am stärksten, wenn die Nachfrage unelastisch ist. 
\item Ein vermehrtes Angebot erhöht Q dann am wenigsten, wenn die Nachfrage unelastisch ist.
\end{itemize}
\subsection{Effizienz und Verteilungsgerechtigkeit auf Wettbewerbsmärkten}
\subsubsection{Effizienzkonzept}
Von {\bf Allokationseffizienz }(oder Effizienz) kann man sprechen, wenn niemand durch eine andere Organisation der Produktion besser gestellt werden kann, ohne dass dadurch zugleich jemand anderer schlechter gestellt wird. Unter den Bedingungen allokativer Effizienz lässt sich eine Steigerung der Bedürfnisbefriedigung oder des Nutzens für eine Person nur durch Schmälerung des Nutzens für eine andere Person erreichen. 

\subsubsection{wirtschaftliche Rente}
\begin{wrapfigure}{r}{0.4\textwidth}
  \begin{center}
  	\vspace{-20pt}
    \includegraphics[width=0.38\textwidth]{img/allokation.jpg}
    \vspace{-20pt}
    \end{center}
\end{wrapfigure}
Die Abbildung zeigt ein neues Konzept, die wirtschaftliche Rente, die als rostfarbener Bereich zwischen den Angebots- und Nachfragekurven im Gleichgewicht dargestellt ist. Die 
wirtschaftliche Rente ist die Summe der im 5. Kapitel beschriebenen Konsumentenrente – des Bereichs zwischen der Nachfragekurve und der Preislinie – und der Produzentenrente – des Bereichs zwischen der Preislinie und der SS-Kurve. Die Produzentenrente beinhaltet die Renten und Gewinne von Firmen und Eigentümern spezialisierter, in der Branche eingesetzter Inputs und gibt den Überschluss der Erträge gegenüber den Produktionskosten an. Die volkswirtschaftliche Rente drückt den aus Produktion und Konsum eines Gutes zusätzlich gewonnenen Nettonutzen oder Wohlstand aus. Sie entspricht der Konsumentenrente plus der Produzentenrente .\\
\\
Eine weitere Möglichkeit, die Effizienz des Wettbewerbsgleichgewichts zu betrachten, ist er Vergleich der wirtschaftlichen Auswirkungen einer kleinen Änderung des Gleichgewichts bei E. Wie der folgende Dreistufenprozess zeigt, ist die Allokation effizient, wenn MU = P = MC.
\begin{itemize}
\item P = MU. Die Konsumenten entscheiden sich für den Kauf von Nahrungsmitteln bis zu jenem Betrag, bei dem gilt: P = MU. Folglich gewinnt jede Person aus der letzten konsumierten Nahrungsmitteleinheit P Nutzeneinheiten oder Utils.
\item P = MC. Als Produzent bietet jede der Personen aus unserem Beispiel Nahrungsmittel bis zu dem Punkt an, an dem der Nahrungsmittelpreis genau den MC der letzten angebotenen Nahrungsmitteleinheit entspricht (wobei hier die MC die Kosten des Freizeitverzichts sind, der zur Produktion der letzten Nahrungsmitteleinheit erforderlich ist). Der Preis entspricht daher den Freizeit-Utils, auf die für die Produktion dieser letzten Nahrungsmitteleinheit verzichtet werden muss.
\item Wenn wir diese beiden Gleichungen zusammenfügen, erkennen wir, dass MU = MC.
Das bedeutet, dass die durch die letzte konsumierte Nahrungsmitteleinheit gewonnenen Nutzeneinheiten exakt den durch den Zeitaufwand für die Produktion dieser letzten produzierten Nahrungsmitteleinheit verlorenen Freizeit-Utils entsprechen. Und genau diese Bedingung, wonach der Grenzgewinn aus der letzten konsumierten Einheit genau den Grenzkosten der Gesellschaft für die letzte produzierte Einheit entspricht, ist es, die uns garantiert, dass ein Wettbewerbsgleichgewicht effizient ist.
\end{itemize}
WICHTIG:\\
Der vollkommene Markt ist ein Instrument zur Verbindung (a) der Bereitschaft der Konsumenten mit entsprechender Kaufkraft, für Güter zu bezahlen, mit (b) den Grenzkosten dieser Güter, die durch das Angebot der Unternehmen repräsentiert werden. Unter bestimmten Bedingungen garantiert der Wettbewerb Effizienz, was bedeutet, dass kein zusätzlicher Nutzen eines Konsumenten erzielbar ist, ohne zugleich den Nutzen eines anderen Konsumenten zu schmälern. Das trifft auch in einer Welt mit unzähligen Produktionsfaktoren und Produkten zu.\\
{\bf Die tragende Rolle der Grenzkosten in einer Marktwirtschaft erklärt sich wie folgt: Nur wenn die Preise den Grenzkosten entsprechen, holt diese Wirtschaft den maximalen Output und die größtmögliche Bedürfnisbe-friedigung aus ihren knappen Ressourcen an Grund und Boden, Arbeit und Kapital heraus. } 


\section{Unvollständiger Wettbewerb und Monopol}
{\bf Unvollständiger Wettbewerb } herrscht in einem Wirtschaftszweig immer dann, wenn einzelne Anbieter ein gewisses Maß an Kontrolle über den Preis ihrer Produkte ausüben.\\
\includegraphics[width=0.88\textwidth]{img/unvollstandig.jpg}
\subsection{Die Formen des unvollständigen Wettbewerbs}
\subsubsection{Marktstrukteren}
\includegraphics[width=0.98\textwidth]{img/wettbewerb.jpg}

\subsubsection{Ursachen für die Unvollkommenheit des Marktes}
Die meisten Fälle unvollständigen Wettbewerbs lassen sich auf zwei wesentliche Gründe zurückführen.
\begin{itemize}
\item[] Erstens gibt es in einem Wirtschaftszweig tendenziell weniger Anbieter, wenn Skaleneffekte und eine damit einhergehende Kostensenkung in der Produktion eine besondere Bedeutung haben. Unter diesen Bedingungen können große Unternehmen einfach billiger produzieren und damit die Kleinen unterbieten, die das häufig nicht überleben.
\item[] Zweitens tendieren Märkte zum unvollständigen Wettbewerb, wenn neue Mitbewerber „Zutrittsbarrieren“ überwinden müssen.
\end{itemize}
\subsubsection{Kosten und Marktunvollkommenheit}
Die entscheidende Frage lautet, welche Bedeutung Skaleneffekte in einem bestimmten Wirtschaftszweig haben. Wenn Skaleneffekte eine wichtige Rolle spielen, werden ein oder mehrere Unternehmen ihre Produktion bis auf ein Niveau erhöhen, bei dem sie schließlich den Großteil der Gesamtproduktionsmenge herstellen. Doch unabhängig davon, wie sich die Branchenstruktur auf Grund von Skaleneffekten darstellt, werden wir in jedem Fall eine Art des unvollständigen Wettbewerbs anstelle einer atomistischen Konkurrenz im  vollständigen Wettbewerb der Preisnehmer und Mengenanpasser vorfinden.
\\
\includegraphics[width=0.98\textwidth]{img/wettbewerb_vergleich.jpg}

\subsubsection{natürliches Monopol}
Ein {\bf natürliches Monopol} ist ein Markt, in dem der Branchenoutput nur von einemeinzigen Unternehmen auf effiziente Weise produziert werden kann. Dieser Fall tritt ein, wenn die Technologie über eine breite Produktionspalette hinweg, die so groß ist wie die gesamte Nachfrage, Skalenvorteile bietet.

\subsubsection{Marktzutrittsbarrieren}
Marktzutrittsbarrieren sind Faktoren, die es neuen Unternehmen schwer machen, in einem Wirtschaftszweig Fuß zu fassen.
\begin{itemize}
\item {\bf Skaleneffekte}
\item {\bf Gesetzliche Beschränkungen.} z.B. Patente, Lizenzen, Importbeschränkungen
\item {\bf Hohe Zutrittskosten } z.B. Planung, Entwicklung und Test eines neuen Flugzeugs
\item {\bf Werbung und Produktdifferenzierung} Die Nachfrage nach jedem der individuell differenzierten Produkte ist so gering, dass es dadurch nur wenigen Unternehmen möglich ist, am tiefsten Punkt ihrer U-förmigen Kostenkurve zu operieren.
\end{itemize} 

\subsection{Grenzerlös und Monopol}
\includegraphics[width=0.98\textwidth]{img/grenzerlos.jpg}

Graphiken:\\
(a) Die rostfarbenen Stufen zeigen die Anstiege des Gesamterlöses aus jeder zusätzlichen Produktionseinheit. MR fällt von Anfang an unter P. MR wird mit unelastischer dd negativ. Eine Glättung der ansteigenden MRStufen ergibt die durchgängige, dünne rostfarbene MR- Kurve, die im Falle der geradlinigen dd immer eine doppelt so hohe Steigung wie dd haben wird.\\
(b) Der Gesamterlös weist eine glockenförmige Kurve auf – Anstieg von null, wobei q = 0, bis zu einem Maximum (wo dd eine Elastizität von 1 aufweist), danach neuerlicher Abfall auf null, wobei P = 0. Die Steigung von TR ergibt die geglätteten MR, ebenso wie der in Sprüngen anwachsende TR die ansteigenden MR-Stufen widerspiegelt.

\subsubsection{Grenzerlös und Preis}
Der Grenzerlös (MR) ist die Erhöhung des Erlöses, die durch eine zusätzlich verkaufte Einheit entsteht. Der MR kann dabei positiv oder negativ sein.\\
Bei abwärtsgerichteter Nachfragekurve gilt:
\begin{equation}
P > MR \text{ ( = P – geringerer Erlös für alle vorherigen q ) } \nonumber
\end{equation}
\subsubsection{Elastizität und Grenzerlös}
\includegraphics[width=0.88\textwidth]{img/grenzerlos_elas.jpg}
\subsubsection{Gewinnmaximierende Bedingungen}
 Definitionsgemäß entspricht der
Gesamtgewinn dem Gesamtertrag abzüglich
der gesamten Kosten; in Symbolen:
\begin{equation}
TP = TR – TC = (P * q) – TC \nonumber
\end{equation}
Daraus leitet sich die wichtige Erkenntnis ab, dass das Gewinnmaximum dann erreicht wird, wenn die Produktionsmenge so groß ist, dass der Grenzerlös des Unternehmens seinen Grenzkosten entspricht.\\
\includegraphics[width=0.98\textwidth]{img/situation.jpg}\\

Das {\bf Gewinnmaximum } eines Monopolisten wird bei jenem Preis (P*) und jener Menge (q*) erreicht, bei denen der Grenzerlös genau den Grenzkosten entspricht:\\ 
MR = MC, bei P* und q* mit dem größtmöglichen Gewinn.\\
\newpage
\begin{wrapfigure}{l}{0.5\textwidth}
  \begin{center}
  	\vspace{-10pt}
    \includegraphics[width=0.48\textwidth]{img/gewinnmaximierung.jpg}
    \vspace{-10pt}
    \end{center}
\end{wrapfigure}
(a) In E, wo die MC- die MR-Kurve schneidet, liegt die Gleichgewichtsposition des Maximalgewinns. Jede Bewegung weg von E geht mit einem Gewinnrückgang einher. Der Preis liegt auf der Nachfragekurve in Punkt G über E; und da P über AC liegt, ist der Maximalgewinn positiv. (Können Sie erklären, warum das rostfarbig unterlegte Rechteck den Gesamtgewinn misst? Und warum die grau unterlegten Dreiecke beiderseits von E den Rückgang des Gesamtgewinns ausweisen, der sich aus einer Abweichung von MR = MC ergäbe?)\\
(b) Dieses Bild enthält die gleiche Aussage bezüglich der Gewinnmaximierung wie (a), verwendet aber die Gesamtkonzepte anstelle der Grenzkonzepte. Die TR-Kurve zeigt den Gesamterlös, während die TC-Kurve die Gesamtkosten darstellt. (Warum nimmt TR bei q = 0 und bei q = 10 den Wert 0 an?) Der Gesamtgewinn (TP) entspricht TR – TC oder geometrisch betrachtet der vertikalen Distanz von TC nach TR. Am Punkt des maximalen Gewinns ist die Differenz zwischen der Gesamterlös- und der Gesamtkostenkurve am größten. Die Steigung jeder Kurve entspricht ihrem Grenzwert (beispielsweise entspricht die Steigung von TR dem MR). Am Gewinnmaximum verlaufen TR und TC parallel und weisen daher eine identische Steigung MR = MC auf.
\subsubsection{Vollständiger Wettbewerb als Grenzfall des unvollständigen Wettbewerbs}
Unter Bedingungen des vollständigen Wettbewerbs entspricht der Preis dem Durchschnittserlös, der seinerseits dem Grenzerlös entspricht (P = MR = AR). Die dd-Kurve und die MR-Kurve eines Unternehmens im vollständigen Wettbewerb fallen als horizontale Linien zusammen. \\
Da ein Unternehmen im vollständigen Wettbewerb jede beliebige Menge zum Marktpreis verkaufen kann, gilt am Punkt des maximalen Gewinns: MR = P = MC

\subsubsection{Marginalprinzip}
Dies ist das Marginalprinzip, das besagt, dass jedermann seine Gewinne oder seine Bedürfnisbefriedigung maximiert, indem er nur Grenzkosten und Grenznutzen einer Entscheidung berücksichtigt. Es gibt unzählige Situationen, in denen sich dieses Marginalprinzip anwenden lässt. Wir haben soeben gesehen, dass das Marginalprinzip des Ausgleichs von Grenzkosten und Grenzerlös die Regel für die Gewinnmaximierung von Unternehmen ist. Ein weiteres Anwendungsbeispiel sind kluge Investitionsentscheidungen. Wenn eine Entscheidung getroffen werden muss, beispielsweise ob man in ein Unternehmen investieren oder ein Haus verkaufen sollte, lohnt es sich immer, frühere Gewinne oder Verluste außer Acht zu lassen und nur aufgrund der Grenzerträge und -kosten zu entscheiden. Das Marginalprinzip ist eine der wichtigsten Lektionen der Volkswirtschaft.


\section{Wie Märkte die Einkommen bestimmen}
\subsection{Einkommen und Vermögen}
\subsubsection{Einkommen}
Mit {\bf Einkommen} sind die Ströme von Einkommen, Zinsen, Dividenden und anderen Werten gemeint, die während eines bestimmten Zeitraums (normalerweise während eines Jahres) auflaufen. Die aggregierten Einkommen in einem Land werden als Volkseinkommen bezeichnet. 
Einkommen ist eine Flussgrösse.
\\
Die Erträge aus einer Marktwirtschaft werden an die Eigentümer der Produktionsfaktoren dieser Wirtschaft in Form von Gehältern, Gewinnen, Mieten und Zinsen verteilt.
\subsubsection{Faktoreinkommen}
{\bf Faktoreinkommen}, beschreibt das Einkommen aus einem spezifischen Gebiet, wie zum Beispiel Arbeit.
\subsubsection{Transferzahlungen}
Regierungen aller Ebenen stellen Einkommen in Form von Transferzahlungen zur Verfügung, worunter staatliche Zahlungen an Einzelpersonen zu verstehen sind, die nicht als Gegenleistung für die Lieferung von Gütern oder Dienstleistungen erfolgen.
\subsubsection{Persönliches Einkommen}
Das persönliche Einkommen besteht aus dem Markteinkommen plus Transferzahlungen. Die meisten Markteinkommen stammen von Löhnen und Gehältern; eine kleine, wohlhabende Minderheit bezieht ihr Markteinkommen aus ihrem Besitz. Die wichtigste Komponente staatlicher Transferleistungen sind Sozialversicherungsprogramme für Senioren. 
\subsubsection{Vermögen}
Mit {\bf Vermögen } bezeichnen wir den Nettogeldwert der Vermögenswerte, die jemand zu einem bestimmten Zeitpunkt besitzt. Vermögen ist eine Bestandsgröße (Kapital, Autos, Häuser, usw.).\\
Die Differenz zwischen Gesamtvermögenswerten und Gesamtverbindlichkeiten wird als Reinvermögen oder Eigenkapital bezeichnet.
\subsection{Grenzproduktivität und Faktorpreise}
Die Theorie der Einkommensverteilung (oder Verteilungstheorie) befasst sich mit der Frage, welche Faktoren die Einkommen in einer Marktwirtschaft bestimmen.\\
Die Nachfrage nach Produktionsfaktoren unterscheidet sich von jener nach Verbrauchsgütern in zweierlei Hinsicht: 
\begin{itemize}
\item Faktornachfragen sind abgeleitete Nachfragen 
\item Faktornachfragen zeichnen sich durch Interdependenz (wechselseitige Abhängigkeit) aus.
\end{itemize} 
Ökonomen bezeichnen die Faktornachfrage als abgeleitete oder derivative Nachfrage. Damit soll ausgedrückt werden, dass ein Unternehmen einen Produktionsfaktor nachfragt, um später damit ein Gut zu  produzieren, das die Verbraucher jetzt oder später nachfragen.
\subsubsection{Verteilungstheorie und Wertgrenzprodukt}
Das {\bf Wertgrenzprodukt (MRP) } stellt den zusätzlichen Erlös dar, den ein Unternehmen durch den Einsatz einer zusätzlichen Faktoreinheit erzielt, wenn die anderen Produktionsfaktoren unverändert belassen werden. Es wird definiert als das Grenzprodukt des jeweiligen Faktors multipliziert mit dem Grenzerlös, der durch den Verkauf einer zusätzlichen Produktionseinheit erzielt wird. Dies gilt ebenso für die Arbeit (L) wie für den Boden (A) und alle anderen Produktionsfaktoren: 
In Symbolen:
\begin{equation}
\text{Wertgrenzprodukt der Arbeit (MRPL)} = MR * MPL \nonumber
\end{equation}
\begin{equation}
\text{Wertgrenzprodukt der Bodens (MRPA)} = MR * MPA \nonumber
\end{equation}
und so weiter. Unter den Bedingungen vollständigen Wettbewerbs gilt, da P = MR:\\
\begin{equation}
\text{Wertgrenzprodukt (MRPi)} = MR * MPi \nonumber
\end{equation}
für jeden Produktionsfaktor.
\subsubsection{Die Nachfrage nach Produktionsfaktoren}
Um Gewinne zu maximieren, müssen Unternehmen solange zusätzliche {\bf Produktionsfaktoren } einsetzen, bis das {\bf Wertgrenzprodukt } des jeweiligen Faktors den Grenzkosten oder dem Preis dieses Faktors entspricht.\\
Die gewinnmaximierende Faktorkombination ist für ein Unternehmen bei vollkommenem Wettbewerb dann erreicht, wenn das Grenzprodukt, multipliziert mit dem Produktpreis, dem Faktorpreis entspricht:\\
\begin{equation}
\text{Grenzprodukt der Arbeit * Produktpreis = Arbeitspreis = Lohn} \nonumber
\end{equation}
\begin{equation}
\text{Grenzprodukt des Bodens * Produktpreis = Bodenpreis = Pacht} \nonumber
\end{equation}
und so weiter.\\ \\
Erklärung:\\
{\it Wir können diese Regel anhand folgender Überlegung verstehen: Nehmen wir an, dass alle Input-Arten in kleine Mengeneinheiten zu je US-\$ 1 unterteilt sind –Arbeitseinheiten zu US-\$ 1, Bodeneinheiten zu US-\$ 1 und so weiter. Um seinen Gewinn zu maximieren, wird das Unternehmen genau bis zu dem Punkt zusätzliche Faktoren kaufen, an dem jedes kleine Paket von US-\$-1-Paket Outputs mit einem Wert von exakt US-\$ 1 produziert. Anders ausgedrückt: Jedes US-\$-1-Faktorpaket produziert MP Einheiten Mais, sodass MP * P genau einen Wert von US-\$ 1 erreicht. Das MRP der US-\$-1-Einheiten beträgt dann bei maximalem Gewinn genau US-\$ 1.}\\

\subsubsection{Minimalkostenregel (Least-Cost-Regel)}
Die Kosten werden minimiert, wenn das Grenzprodukt pro Geldeinheit Input für alle Produktionsfaktoren gleich hoch ist. Diese Regel gilt für Unternehmen auf Produkt- oder Gütermärkten mit vollständigem Wettbewerb ebenso wie auf Produktmärkten mit unvollständigem Wettbewerb.

\subsubsection{Substitutionsregel}
Eine logische Folge aus
der Least-Cost-Regel bildet die Substitutionsregel. Wenn der Preis eines Produktionsfaktors steigt, während die anderen Faktorpreise konstant bleiben, entsteht dem Unternehmen ein Vorteil, wenn es den teureren Faktor durch eine größere Menge der anderen Faktoren substituiert. 

\subsection{Angebot an Produktionsfaktoren}
Eine vollständige Analyse der Bestimmungsgrößen von Faktorpreisen und Einkommen umfasst nicht nur die soeben beschriebene Faktornachfrage, sondern auch das Angebot an den  diversen Produktionsfaktoren.\\
Lässt sich irgendetwas über die Elastizität des Faktorangebotes aussagen? Tatsächlich kann die Angebotskurve einen positiven Anstieg aufweisen, sie kann vertikal verlaufen oder sogar eine negative Steigung haben. Bei den meisten Faktoren ist zu erwarten, dass das Angebot auf steigende Faktorpreise langfristig positiv reagiert. In diesem Fall wäre die Angebotskurve nach rechts oben gerichtet. Normalerweise ist weiterhin davon auszugehen, dass das gesamte Angebot an Grund und Boden vom Preis nicht beeinflusst wird; in diesem Fall ist das gesamte Bodenangebot demnach absolut unelastisch, was bedeutet, dass die Angebotskurve vertikal verläuft.  
\subsection{Faktorpreisbildung durch Angebot und Nachfrage}
Der Gleichgewichtspreis eines Produktionsfaktors auf einem vollkommenen Markt stellt sich auf jenem Niveau ein, auf dem die angebotenen und die nachgefragten Mengen identisch sind.\\
{\it z.B. Wir addieren alle Nachfragekurven der einzelnen Unternehmen horizontal, um die Marktnachfragekurve für Boden zu erhalten.}\\

\begin{wrapfigure}{l}{0.5\textwidth}
  \begin{center}
  	\vspace{-10pt}
    \includegraphics[width=0.48\textwidth]{img/faktor-arbeit.jpg}
    \vspace{-10pt}
    \end{center}
\end{wrapfigure}
In (a) sehen wir die Auswirkungen eines begrenzten Angebots an Chirurgen: geringe Zahl und hohe Einkünfte der
Chirurgen. Welche Auswirkungen auf die Gesamteinkünfte von Chirurgen und den Preis einer Operation hätte wohl eine
alternde Bevölkerung, die eine erhöhte Nachfrage nach Chirurgen bewirkt?
In (b) ermöglichen der offene Marktzutritt und die geringen Ausbildungserfordernisse ein besonders elastisches Angebot
von Fast-Food-Arbeitern. Die Löhne werden gedrückt, und der Beschäftigungsstand ist hoch. Welche Auswirkungen auf
Löhne und Beschäftigungsstand würden sich bei vermehrtem Zustrom von Teenagern in diese Jobs ergeben?
\subsection{Verteilung des Volkseinkommens}
\begin{wrapfigure}{r}{0.3\textwidth}
  \begin{center}
  	\vspace{-10pt}
    \includegraphics[width=0.28\textwidth]{img/bodenrente.jpg}
    \vspace{-10pt}
    \end{center}
\end{wrapfigure}
Ein
erster Arbeiter erwirtschaftet ein hohes Grenzprodukt, weil er eine sehr große Menge an Boden zu bearbeiten hat. Das Grenzprodukt von Arbeiter 2 ist bereits kleiner. Doch die beiden Arbeiter sind gleichwertig, sie müssen also denselben Lohn bekommen. Die Frage ist nur, welchen? Das MP von Arbeiter 1, das von Arbeiter 2 oder den Durchschnitt aus diesen beiden Werten?\\ 
Unter den Bedingungen des vollständigen Wettbewerbs ist die Antwort eindeutig: Der Grundeigentümer wird keinen Arbeiter einstellen, wenn der Marktlohn, den er ihm zahlen müsste, über dem Grenzprodukt dieses Arbeiters liegt. So sorgt der Wettbewerb dafür, dass alle Arbeiter einen Lohn erhalten, der dem Grenzprodukt des letzten Arbeiters entspricht.\\
Graphik:\\
Jeder vertikale Streifen stellt das Grenzprodukt der jeweiligen Arbeitseinheit dar. Die gesamte nationale Produktionsleistung 0DES wird ermittelt, indem alle vertikalen Streifen von MP bis zum Arbeitsgesamtangebot in S addiert werden. Die Produktionsverteilung leitet sich aus der Grenzproduktkurve ab. Die Gesamtlöhne bilden das untere Rechteck (entsprechend Lohn 0N mal Arbeitsmenge 0S). Das verbleibende Dreieck NDE entspricht der Bodenrente.
\subsubsection{Grenzproduktivitätstheorie bei einer Vielzahl von Produktionsfaktoren}
Auf vollkommenen Märkten wird die Inputnachfrage durch die Grenzprodukte der Produktionsfaktoren bestimmt. Im vereinfachten Fall, in dem die Faktoren in Form des einzigen Outputs bezahlt werden, erhalten wir folgende Gleichung:
\\
Lohn = Grenzprodukt der Arbeit\\
Rente = Grenzprodukt des Bodens\\
und so weiter für jeden Produktionsfaktor.
Damit werden genau 100 Prozent des Produktionsergebnisses bzw. der produzierten
Menge unter allen Produktionsfaktoren verteilt, nicht mehr und nicht weniger.
\section{Arbeitsmarkt}
\subsection{Die Grundlagen der Lohnbildung}
In der Lohnanalyse tendieren Ökonomen dazu, den durchschnittlichen Reallohn heranzuziehen, der die Kaufkraft eines Stundenlohns oder den Lohn, dividiert durch die Lebenshaltungskosten, darstellt. 
\subsection{Die Arbeitsnachfrage}
\subsubsection{Unterschiede in der Grenzproduktivität}
Wir beginnen unsere Untersuchung des allgemeinen Lohnniveaus, indem wir die für die Arbeitsnachfrage bestimmenden Faktoren analysieren.\\
{\bf Grenzproduktivität}\\
Erstens steigt die Grenzproduktivität, wenn Arbeitnehmer über mehr oder bessere Kapitalgüter verfügen, mit denen sie arbeiten können. Und zweitens ist die Grenzproduktivität gut aus- und weitergebildeter Arbeitnehmer im Allgemeinen höher als jene von schlecht ausgebildetem „Humankapital“. \\
{\bf Qualität der Arbeit}\\
Die Qualität des Faktors Arbeit ist der zweite Faktor, der sich auf das allgemeine Lohnniveau auswirkt. Welchen Maßstab man auch immer heranzieht – Alphabetisierung, Schul- oder Berufsbildung –, die US-Arbeitskräfte des Jahres 2000 sind denen von 1900 haushoch überlegen. \\ \\
{\it Das durchschnittliche Ausbildungsniveau in Mexiko liegt weit unter jenem der USA,und ein gar nicht so kleiner Prozentsatz der Bevölkerung kann nach wie vor nicht lesen und schreiben. Verglichen mit den USA verfügt ein Land wie Mexiko über sehr viel weniger Kapital, mit dem sich arbeiten lässt. Viele der Straßen Mexikos sind staubige Pisten, es gibt nur wenige Computer und Faxgeräte, und ein Großteil der maschinellen Ausstattung ist alt und befindet sich in einem schlechten Zustand. Alle diese Faktoren senken die Grenzproduktivität der Arbeit und drücken auf die Löhne.}
\subsection{Das Arbeitsangebot}
Das Arbeitsangebot entspricht der Arbeitszeit, die die Bevölkerung mit verschiedenen Erwerbstätigkeiten zubringen möchte. Die drei wesentlichen Einflussfaktoren auf das Arbeitsangebot sind Arbeitszeit, Erwerbsquote und Zuwanderung.\\
\subsubsection{Arbeitszeit}
Versetzen Sie sich in die Lage eines Arbeiters, dem gerade ein höherer Stundenlohn angeboten wurde und der selbst entscheiden kann, wie lange er arbeiten möchte. In dieser Situation fühlt er sich hin- und hergerissen. Auf der einen Seite macht sich der {\bf Substitutionseffekt}  bemerkbar. (Die Wirkung des Substitutionseffekts haben wir bereits in Kapitel 5 erläutert: Die Konsumenten tendieren dazu, mehr von einem billigeren Ersatz- oder Substitutionsgut und weniger von einem anderen, verteuerten Gut zu kaufen.) Da nun jede Arbeitsstunde besser bezahlt wird, ist auch jede Stunde Freizeit teurer geworden. Es entsteht somit ein Anreiz, Freizeit durch zusätzliche Arbeit zu ersetzen. Dem Substitutionseffekt entgegen wirkt jedoch der {\bf Einkommenseffekt}: Je höher der Stundenlohn, desto höher auch das Gesamteinkommen. Ein höheres Einkommen weckt den Wunsch, mehr Güter und Dienstleistungen zu kaufen, und löst ein Bedürfnis nach mehr Freizeit aus. Mit höheren Löhnen kann man länger Urlaub machen oder früher in Pension gehen, als das bei niedrigen Löhnen der Fall wäre.
\subsubsection{Erwerbsquote}
Um eine so tiefgreifende Verschiebung der Arbeitsmuster erklären zu können, muss man über die wirtschaftlichen Aspekte hinaus auch soziale Prozesse, die zu einer neuen Einstellung gegenüber der Rolle der Frauen als Mütter, Hausfrauen und Arbeitskräfte geführt haben, berücksichtigen.
\subsubsection{Zuwanderung}
Aus der Perspektive des Arbeitsangebotes ist durch die jüngsten Immigrationswellen ein Überangebot ungeschulter gegenüber gut ausgebildeten Arbeitskräften in den USA entstanden. Studien zufolge hat diese angebotsseitige Veränderung zu einem Rückgang der Löhne schlecht ausgebildeter Arbeitskräfte gegenüber College-Absolventen geführt.

\subsection{Lohnunterschiede}
\subsubsection{Kompensatorische Unterschiede}
Unterschiede in der Entlohnung, die dazu dienen, die jeweilige Attraktivität oder nichtmonetäre Unterschiede zwischen verschiedenen Jobs auszugleichen, werden als kompensatorische Lohnunterschiede (oder kompensatorisches Lohndifferenzial) bezeichnet. 

\subsubsection{Arbeitsqualität}
Ein Schlüssel zur Erklärung von Lohnunterschieden liegt in den enormen qualitativen Unterschieden zwischen den Arbeitskräften, die auf verschiedene angeborene geistige und körperliche Fähigkeiten, auf Erziehung, Schul- und Berufsausbildung sowie auf unterschiedliche Berufserfahrung zurückzuführen sind. \\
{\it Der Begriff „Humankapital“ bezieht sich auf den Bestand nützlicher und wertvoller Kenntnisse, die von Leuten während ihrer Schul- und Ausbildungszeit akkumuliert werden.} 

\subsubsection{Lohn für einzigartige Begabungen}
Gewisse Menschen verfügen über eine spezielle Begabung, die im heutigen Wirtschaftsleben offensichtlich geschätzt wird. Außerhalb ihres Fachgebietes würden sie wahrscheinlich nur einen winzigen Bruchteil ihres Einkommens erzielen.\\ \\
Ökonomen bezeichnen jenen Teil des Lohnes, der über die beste Alternativbeschäftigung hinaus bezahlt wird, als reine volkswirtschaftliche Rente; diese zusätzlichen Einkünfte bilden das logische Äquivalent zu den aus Grund und Boden erzielten Renten.\\
\\
\includegraphics[width=0.78\textwidth]{img/arbeit.jpg}


\subsection{Probleme und politische Massnahmen}
Eine der Quellen des unvollständigen Wettbewerbs sind die Gewerkschaften. Sie repräsentieren einen erheblichen, wenn auch schrumpfenden Anteil der Arbeitnehmer. Eine zweite wichtige Facette des Arbeitsmarktes ist die Diskriminierung – ebenfalls weniger bedeutend als in früheren Jahrzehnten, aber immer noch ein Problem, das es zu bedenken gilt. Ein dritter Faktor mit Einfluss auf die Arbeitsmärkte sind staatliche Eingriffe. \\
\subsubsection{Gewerkschaften}
Eine Gewerkschaft erwirbt Marktmacht, indem ihr ein legales Monopol für die Bereitstellung von Arbeitsleistungen an ein bestimmtes Unternehmen oder eine bestimmte Branche eingeräumt wird.\\
Löhne und Sozialleistungen gewerkschaftlich organisierter Arbeitnehmer werden in {\bf Tarifverträgen} festgelegt. Kernstück eines Kollektivvertrages ist das {\bf Tarifpaket}. Es legt die Höhe des Basislohns für verschiedene Jobkategorien sowie Richtlinien für Urlaube und Arbeitspausen fest. Ein zweiter wichtiger und häufig auch umstrittener Teil des Kollektivvertrages betrifft die {\bf Regelung der Arbeitsbedingungen}. Er bezieht sich auf Arbeitsanweisungen und die Zuteilung von Aufgabenbereichen, die Arbeitssicherheit und die zulässige Arbeitsbelastung. \\ \\
{\bf Auswirkungen auf Löhne der Beschäftigten und Baschäftigung}\\
Die Analyseergebnisse lassen darauf schließen, dass Gewerkschaften in ihren Lohnverhandlungen dort am erfolgreichsten sind, wo sie (wie beispielsweise in der Autoindustrie) das Arbeitsangebot monopolisieren und den Zutritt zum Arbeitsmarkt effektiv kontrollieren können. Erfahrungen im gewerkschaftlich stark organisierten Europa legen den Schluss nahe, dass Gewerkschaften, so es ihnen gelingt, das allgemeine Lohnniveau anzuheben, manchmal eine inflationstreibende Lohn-Preis-Spirale in Gang setzen, sodass auch hier kaum bleibende Auswirkungen auf die Reallöhne festzustellen sind.\\ Wenn nun die Gewerkschaften das allgemeine Reallohnniveau nicht nachhaltig beeinflussen können, so schließen wir daraus, dass sie vor allem auf die relativen Löhne einwirken können. Dies bedeutet, dass die Löhne in gewerkschaftlich organisierten Branchen im Verhältnis zu jenen in nicht organisierten Bereichen ansteigen. Außerdem kommt es zu einer geringeren Beschäftigung in den organisierten und zu höheren Beschäftigungszahlen in den nicht organisierten Branchen.

\subsubsection{Klassische und konjunkturell bedingte Arbeitslosigkeit}
Als {\bf klassische Arbeitslosigkeit } wird diese Situation deshalb bezeichnet, weil sie aus Löhnen resultiert, die über das Wettbewerbsniveau angehoben wurden. Ökonomen stellen im Allgemeinen die klassische Arbeitslosigkeit der {\bf konjunkturell bedingten Arbeitslosigkeit} gegenüber, die als Keynesianische Arbeitslosigkeit bezeichnet wird und sich aus einer unzureichenden Gesamtnachfrage ergibt. 


{\it Unsere Analyse führt uns zu dem Schluss, dass ein Land, in dem überhöhte Reallöhne bezahlt werden, mit einer höheren Arbeitslosigkeit zu rechnen hat. Diese Arbeitslosigkeit reagiert nicht auf die traditionellen makroökonomischen Maßnahmen, also auf eine Erhöhung der gesamtwirtschaftlichen Ausgaben, sondern erfordert konkrete Schritte zur Senkung der Reallöhne.}


\subsubsection{Diskriminierung}
Wenn aufgrund irrelevanter persönlicher Merkmale wie Rasse, Geschlecht, sexueller Orientierung oder Religion ökonomische Unterschiede auftreten, sprechen wir von Diskriminierung. Diskriminierung bedeutet im typischen Fall entweder (a) eine unterschiedliche Behandlung von Menschen aufgrund persönlicher Merkmale oder (b) Praktiken (beispielsweise Tests), die sich auf bestimmte Gruppen nachteilig auswirken.

\subsubsection{Statistische Diskriminierung}
Wir sprechen von statistischer Diskriminierung, wenn Individuen aufgrund des verbreiteten Verhaltens von Mitgliedern ihrer Gruppe, nicht aber anhand ihrer persönlichen Merkmale und ihres Verhaltens beurteilt und behandelt werden.\\
Statistische Diskriminierung führt zu Ineffizienzen, weil sie Stereotype verstärkt und die Anreize einzelner Mitglieder einer Gruppe verringert, sich weiterzubilden und Erfahrungen zu sammeln.\\
\newpage
\section{Steuern und Staatsausgaben}
\subsection{Staatliche Eingriffe in die Wirtschaft}
\begin{itemize}
\item Steuern auf Einkommen sowie auf Güter und Dienstleistungen
\item Öffentliche Ausgaben für bestimmte Güter und Dienstleistungen
\item Regulierungs- oder staatliche Aufsichtsmaßnahmen, die die Staatsbürger dazu anhalten, bestimmte wirtschaftliche Aktivitäten auszuüben oder zu unterlassen \end{itemize}
{\bf Staatsquote}: der Anteil der öffentlichen Ausgaben aller Verwaltungsebenen am
Bruttoinlandsprodukt
\subsubsection{Die Zunahme staatlicher Eingriffe und Regulierungsmaßnahmen}
Die amerikanische Wirtschaft des 19. Jahrhunderts kam von allen modernen Gesellschaften dem Extrem der reinen Laissezfaire-Gesellschaft am nächsten, jenem System, von dem der britische Historiker Thomas Carlyle gesagt hat, es sei nichts weiter als eine Kombination aus „Anarchie und Polizei“. Diese politische Philosophie ließ den Menschen große persönliche Freiheit bei der Verfolgung ihrer wirtschaftlichen Interessen und führte zu einem Jahrhundert rapiden materiellen Fortschritts. Doch Kritiker sahen in diesem Laissez-faire-Idyll gravierende Mängel. So berichten Historiker über periodisch auftretende Wirtschaftskrisen, extreme Armut und Ungleichheit, tief sitzende rassische Diskriminierung und die Vergiftung von Wasser, Boden und Luft durch rücksichtslose Umweltverschmutzung.

\subsection{Funktionen des Staates}
\subsubsection{Verbesserung der volkswirtschaftlichen Effizienz}
Ein zentrales wirtschaftspolitisches Ziel des Staates besteht darin, auf eine sozial wünschenswerte Ressourcenallokation hinzuwirken.
\begin{itemize}
\item {\bf Grenzen der unsichtbaren Hand} \\
Selbst in dem hypothetischen Fall wenn es bei funktionierendem Preismechanismus eine Arbeitsteilung zwischen Menschen und Regionen gäbe, käme dem Staat eine bedeutende Rolle zu. Gerichte und Polizei wären weiterhin nötig, um über die Erfüllung von Verträgen zu wachen, Betrug und Gewalt zu vereiteln, Diebstähle und Aggressionen von außen zu verhindern und die gesetzlich gewährleisteten Eigentumsrechte zu schützen.
\item {\bf Unvermeidliche Interdependenzen} 
Man könnte auch sagen, der Staat setzt seine Waffen häufig ein, um die folgenden Formen von gravierendem Marktversagen zu korrigieren:
\begin{itemize}
\item Zusammenbruch des vollständigen Wettbewerbs
\item Externalitäten und öffentliche Güter. Ein unregulierter Markt produziert möglicherweise zu viel Luftverschmutzung und investiert zu wenig in Volksgesundheit und Bildung 
\item Unzureichende Information. Unregulierte Märkte bieten Konsumenten oft nur mangelhafte Informationen, die nicht ausreichen, um gut informiert Entscheidungen treffen zu können
\end{itemize}
\end{itemize}

\subsubsection{Verringerung der wirtschaftlichen Ungleichheit}
Auch wenn die unsichtbare Hand tatsächlich funktioniert und enorm effizient ist, kann siedurchaus eine sehr ungleichmäßige Einkommensverteilung bewirken. Unter Laissez-faire- edingungen wird der Reichtum oder die Armut der Menschen durch ihren Geburtsort, ihr ererbtes Vermögen, ihre Talente und ihren Fleiß, durch das Glück, das sie bei Ölbohrungenhaben, oder durch ihr Geschlecht oder ihre Hautfarbe bestimmt.  

\subsubsection{Stabilisierung der Wirtschaft durch wirtschaftspolitische Maßnahmen}
Heute ist der Staat dafür verantwortlich, dass sich eine wirtschaftliche Krisensituation mit derart weitreichenden negativen Folgen nicht wiederholt. Als Instrumente stehen ihm dazu eine entsprechende Geld- und Fiskalpolitik sowie eine strikte Regulierung des Finanzsystems zur Verfügung.

\subsubsection{Formulierung und Umsetzung internationaler Wirtschaftspolitik}
Der Staat ist inzwischen ein maßgeblicher Vertreter der nationalen Interessen auf der internationalen Bühne sowie bei der Aushandlung von Verträgen mit anderen Ländern über verschiedenste Fragestellungen. Dabei lassen sich die wirtschaftspolitischen Themen in vier Hauptgebiete einteilen:
\begin{itemize}
\item {\bf Abbau von Handelsbeschränkungen}\\
Ein wesentlicher Bereich der Wirtschaftspolitik beschäftigt sich mit der Harmonisierung von Gesetzen und mit dem Abbau von Handelsschranken, um eine vorteilhafte internationale Spezialisierung und Arbeitsteilung zu fördern. 
\item {\bf Koordinierung makroökonomischer Maßnahmen} \\
Die einzelnen Staaten haben erkannt, dass die Fiskal- und Geldpolitik anderer Staatensich  auf ihre heimische Inflation, Arbeitsmarktlage und Finanzsituation auswirkt. Dasinternationale Finanzsystem kann sich nicht selbst steuern; die Schaffung eines reibungslos funktionierenden Wechselkurssystems ist eine Voraussetzung für einen effizienten internationalen Handel 
\item {\bf Internationaler Umweltschutz}\\
Die neueste Facette internationaler Wirtschaftspolitik ist die Kooperation von Staaten auf ökologischem Gebiet, wenn einzelne Länder Umweltprobleme verursachen oder wenn Staaten von den Umweltproblemen anderer betroffen sind.  
\end{itemize}


\subsubsection{Public-Choice Theorie}
Bisher haben sich unsere Analysen weitgehend auf die normative Theorie des Staates konzentriert – auf die Politik, die Staaten umsetzen sollten, um das Wohlergehen ihrer Bevölkerungen zu erhöhen. Doch die Ökonomen hängen heute hinsichtlich des Marktes oder auch des Staates keinen Illusionen mehr an. Regierungen können durchaus auch schlechte Entscheidungen treffen oder gute Ideen schlecht umsetzen. Ebenso wie es ein Versagen des Marktes gibt – etwa aufgrund von Monopolen oder Umweltverschmutzung –, kann auch der Staat „versagen“, wenn staatliche Eingriffe zu Verschwendung oderzu einer unerwünschten  Einkommensumverteilung führen. Mit diesen Themenbereichen beschäftigt sich die Public- hoice-Theorie (Theorie der öffentlichen Entscheidung), also jener Zweig der Volks- und Politikwissenschaft, der erforscht, wie die Entscheidungsstrukturen in Staaten beschaffen sind. Die Public-ChoiceTheorie untersucht, wie verschiedene Abstimmungsmechanismen funktionieren können, und weist nach, dass es keine idealen Mechanismen gibt, die die individuellen Präferenzen zu gesamtgesellschaftlichen Entscheidungen summieren. 

\subsection{Staatsausgaben}
\subsubsection{Fiskalföderalismus}
Ein effizientes System des Fiskalföderalismus berücksichtigt, wie der Nutzen aus öffentlichen Programmen über politische Grenzen hinweg spürbar wird. Am effizientesten ist es, die Steuer- und Ausgabenentscheidungen so zu treffen, dass die Nutznießer der Programme die Steuern bezahlen und die Vor- und Nachteile der jeweiligen Maßnahme gegeneinander abwägen können.

\subsection{Volkswirtschaftliche Aspekte des Steuerwesens}
Mit seinen Steuergesetzen entscheidet der Staat letztlich darüber, wie die benötigten Ressourcen von den Haushalten und Unternehmen für öffentliche Zwecke abgezogenwerden können. Die durch die Besteuerung aufgebrachten Gelder sind das Vehikel, mit dem reale Ressourcen von privaten zu öffentlichen Gütern verlagert werden.\\

\subsubsection{Steuerprinzipien}
\begin{wrapfigure}{r}{0.3\textwidth}
  \begin{center}
  	\vspace{-10pt}
    \includegraphics[width=0.28\textwidth]{img/steuer.jpg}
    \vspace{-10pt}
    \end{center}
\end{wrapfigure}
Das Prinzip der {\bf horizontalen Gerechtigkeit} besagt, dass Gleiche gleich behandelt werden sollten. Der {\bf vertikalen Gerechtigkeit} zufolge sollten Menschen mit ungleichen Lebensumständen nicht gleich, aber gerecht behandelt werden. Es gibt aber keine Übereinstimmung darüber, wie eine vertikale Gerechtigkeit in der Realität aussehen sollte.\\
Ein {\bf Steuersystem} wird als proportional, progressiv oder regressiv bezeichnet, je nachdem, ob von Steuerpflichtigen mit hohem Einkommen derselbe, ein höherer oder ein geringerer Einkommensprozentsatz erhoben wird als von Beziehern niedriger Einkommen.\\
Der {\bf Grenzsteuersatz } ist ein zentrales Konzept der Steueranalyse. Er bezieht sich auf die pro zusätzlicher Einkommenseinheit zusätzlich bezahlte Steuer und ist besonders wichtig, um die Anreizwirkung der Besteuerung zu verstehen. 

\section{Wie werden Märkte effizienter?}
\subsection{Staatliche Regulierungsmassnahmen in Theorie und Praxis}
\subsubsection{Zwei Arten der Regulierung}
Regulierung bedeutet staatliche Verordnungen oder Gesetze, die zur Kontrolle der Preis-,Verkaufs- oder Produktionsentscheidungen  von Unternehmen erlassen werden.
\begin{itemize}
\item Die {\bf wirtschaftliche Regulierung } bezieht sich auf die Kontrolle vonPreisen, Marktzutritts- und -austrittsbedingungen sowie auf Produktstandards in bestimmten Branchen. Am wichtigsten ist dieser Ansatz in Branchen mit einem natürlichen Monopol.  
\item Die {\bf soziale Regulierung}, die eingesetzt wird, um unsere Umwelt oder die Gesundheit und Sicherheit von Arbeitnehmern und Konsumenten zu schützen. 
\end{itemize}
\subsubsection{Begründung für Regulierungen}
\begin{itemize}
\item {\bf Einschränkung der Marktmacht} \\
Traditionell werden staatliche Regulierungsmaßnahmen normativ gesehen: Regulierungsmaßnahmen sollen dazu dienen, schwerwiegende Formen des Marktversagens zu korrigieren. Oder konkreter, der Staat sollte jene Wirtschaftszweige regulieren, in denen zu wenige Unternehmen tätig sind, um eine gesunde Konkurrenz zu gewährleisten. Wirtschaftszweige bedürfen speziell im Fall eines natürlichen Monopols der staatlichen Regulierung. 
\item {\bf Behebung von Informationsversagen} \\
Ein weiterer Grund für staatliche Regulierungsmaßnahmen besteht in der unzureichenden Information der Konsumenten über die verfügbaren Produkte
\item {\bf Externe Effekte} \\
Auch beim Auftreten externer Effekte können staatliche Regulierungsmaßnahmen gerechtfertigt sein. Das klassische Beispiel einer Regulierung dieses Typs, sind Maßnahmen zum Schutz der Umwelt. Doch es gibt auch noch weitere interessante Fälle. Zu nennen wären etwa raumplanerische Aktivitäten und Flächennutzungspläne, die den Einsatz von Grund und Boden durch die Eigentümer regeln.
\end{itemize}
\subsubsection{Art der Regulierung}
Üblicherweise wird im Zuge staatlicher Regulierung den Unternehmen ein Durchschnittspreis vorgeschrieben.
\begin{itemize}
\item {\bf Ideale Preisregulierung}\\
Wenn P = MC die Idealformel schlechthin ist, stellt sich die Frage, warum die Aufsichtsbehörden den Monopolisten nicht einfach zwingen, seinen Preis so weit zu senken, bis er den Grenzkosten am Schnittpunkt von DD und MC entspricht (in I). Tatsächlich ist die Gleichung P = MC oder die Preisbildung in Höhe der Grenzkosten das Idealziel im Sinne volkswirtschaftlicher Effizienz. Doch in der Praxis ist die Sache nicht so einfach: Setzt ein Unternehmen mit sinkenden Durchschnittskosten seine Preise auf Höhe der Grenzkosten fest, so entsteht ihm dadurch ein chronischer Verlust. Der Grund liegt darin, dass bei sinkenden Durchschnittskosten AC gilt: MC < AC, sodass die Gleichsetzung P = MC impliziert, dass der Preis unter die Durchschnittskosten fällt (P < AC). Liegt der Preis (oder liegen die durchschnittlichen Erträge) unter den Durchschnittskosten, verliert das Unternehmen Geld. 
\item {\bf Zusammengesetzte Preise oder Gebühren}\\
Das Unternehmen berechnet dabei eine fixe Gebühr (beispielsweise einige US-Dollarpro Monat), um seine Betriebskosten abdecken zu können, und zusätzlich einen Aufschlag für die variablen Kosten (etwa pro telefonierter Einheit, pro Kilowattstunde Strom oder dergleichen), um seine Grenzkosten decken zu können. Mit diesem Ansatz käme man den idealen Grenzkostenpreisen noch näher als mit der traditionellen Preisgestaltung anhand der Durchschnittskosten.
\end{itemize}
\subsubsection{Kosten der Regulierung}
Sudien kommen zum Schluss, dass sich eine ökonomische Regulierung hauptsächlich in Effizienzverlusten und einer massiven Einkommensumverteilung äußert. Die Erfolgsbilanz der sozialen Regulierung ist dagegen gemischt. In einigen Fällen ergeben sich signifikante Vorteile, während andere kostenintensiv, aber nicht besonders nützlich sind.  

\subsection{Fusionen}
\begin{itemize}
\item {\bf Horizontale Fusionen}, bei denen sich Unternehmen desselben Wirtschaftszweiges verbinden, sind verboten,wenn die Fusion den  Wettbewerb im betreffenden Sektor merklich hemmt. 
\item {\bf Vertikale Fusionen} sind Zusammenschlüsse zweier Unternehmen, in denen verschiedene Stufen desselben Produktionsprozesses stattfinden. In den letzten Jahren griffen amerikanische Gerichte häufig hart gegen vertikale Fusionen durch. Ihre Sorge galt dabei den potenziellen Wettbewerbsbeschränkungen durch Exklusivverträge, wenn sich zwei unabhängige Unternehmen auf diese Weise zusammentaten.
\item Eine dritte Art der Unternehmenszusammenführung, die als {\bf konglomerater Zusammenschluss}  bezeichnet wird, bezieht sich auf bis dahin unverbundene Unternehmen. Im Zuge eines konglomeraten Zusammenschlusses kann etwa eine Stahl- oder Chemiegesellschaft eine Ölgesellschaft aufkaufen.
\end{itemize}

\section{Makroökonomie}
\subsection{Ziele}
Die wichtigsten makroökonomischen Ziele sind eine rasche und beträchtliche Steigerung der Produktion, niedrige Arbeitslosigkeit und ein stabiles Preisniveau.
\subsubsection{Prodkutionsleistung}
Das umfassendste Maß für die Gesamtleistung einer Volkswirtschaft ist das Bruttoinlandsprodukt (BIP). Das BIP ist das Maß für den Marktwert aller Endprodukte und Dienstleistungen – Bier, Autos, Rockkonzerte, Eselsritte und so weiter –, die ein Land innerhalb eines Jahres produziert oder bereitstellt.
\begin{itemize}
\item Das {\bf nominale BIP} wird anhand der tatsächlichen Marktpreise gemessen.
\item Das {\bf reale BIP }lässt sich anhand konstanter oder nicht variabler Preise errechnen (dabei multiplizieren wir beispielsweise die Anzahl der produzierten Autos mit den Autopreisen des Jahres 2000) Kein anderes Maß für die Gesamtleistung wird so genau beobachtet wie das reale BIP; an diesem ständig beobachteten Pulsschlag wird die Gesundheit der Wirtschaft gemessen. \\
Trotz der kurzfristigen Schwankungen in den Konjunkturzyklen zeigen reife Wirtschaften im Allgemeinen ein beständiges langfristiges Wachstum des realen BIP und eine Verbesserung des Lebensstandards; dieser Prozess wird als {\it Wirtschaftswachstum } bezeichnet. 
\item Das {\bf potenzielle BIP } stellt das höchste nachhaltige Produktionsniveau dar, das eine Volkswirtschaft erreichen kann. Schöpft eine Wirtschaft ihr Produktionspotenzial aus, werden die Arbeitskräfte und der Kapitalstock weitestgehend genutzt. Wenn die tatsächliche Produktionsleistung über das Produktionspotenzial steigt, nimmt in der Regel die Inflation zu, während eine Produktionsleistung unterhalb des Produktionspotenzials zu Arbeitslosigkeit führt.
\item Eine Rezession ist eine Zeit, in der die gesamte Produktionsleistung sowie Einkommen und Beschäftigung deutlich zurückgehen; sie dauert in der Regelzwischen sechs Monaten und einem Jahr und ist durch ein deutliches Schrumpfen zahlreicher Wirtschaftsbereiche gekennzeichnet. Einen lang anhaltenden und stark ausgeprägten
Abschwung bezeichnet man als Depression.
\end{itemize}
\subsubsection{Arbeitslosenquote}
Die Arbeitslosenquote zeigt üblicherweise an, wo wir uns innerhalb eines Konjunkturzyklus befinden.
\subsubsection{Preisstabilität}
Preisstabilität. Das dritte wirtschaftspolitische Ziel ist die Erhaltung der Preisstabilität. Dieser Ausdruck bedeutet, dass das Gesamtpreisniveau entweder stabil ist oder nur langsam steigt. Um die Preisentwicklung zu verfolgen, arbeiten Statistiker mit Preisindizes, Instrumenten zur Messung des Gesamtpreisniveaus. Ein wichtiges Beispiel hierfür ist der Verbraucherpreisindex (VPI), der den Durchschnittspreis von Waren und Dienstleistungen misst, die Konsumenten kaufen. Das gesamtwirtschaftliche Preisniveau wird häufig durch den Buchstaben P abgekürzt. Wirtschaftswissenschaftler messen die Preisstabilität anhand der Inflationsrate. Die Inflationsrate ist die prozentuale Änderung des gesamtwirtschaftlichen Preisniveaus von einem Jahr zum nächsten. Beispielsweise lag der Verbraucherpreisindex 2001 bei 177,1 und bei 179,9 im Jahre 2002 (wobei das Jahr 1983 = 100 gesetzt wurde). 
\subsection{Instrumente}
Einem Staat stehen zur Erreichung makroökonomischer Ziele zwei wesentliche Instrumente zur Verfügung – die Fiskal- und die Geldpolitik.
\subsubsection{Fiskalpolitik}
Zur Fiskalpolitik gehören Staatsausgaben und Steuern. Staatsausgaben beeinflussen das relative Ausmaß kollektiver Ausgaben im Vergleich zum privaten Konsum. Durch die Erhebung von Steuern verringern sich die Einkommen, die privaten Ausgaben sinken und die privaten Ersparnisse werden berührt. Außerdem ergeben sich Auswirkungen auf Investitionen und die potenzielle Produktionsleistung. Die Fiskalpolitik wird heute hauptsächlich eingesetzt, um durch die Beeinflussung der Ersparnisbildung sowie der Anreize zu arbeiten und zu sparen das langfristige Wirtschaftswachstum zu steuern. 
\subsubsection{Geldpolitik}
Durch die Geldpolitik der Zentralbank werden die Geldmenge und die Finanzkonditionen bestimmt. Veränderungen der verfügbaren Geldmenge führen zu einer Erhöhung oder Senkung der Zinssätze und haben Auswirkungen auf die Ausgaben in verschiedenen Bereichen, beispielsweise für Investitionen, für Wohnungen und für den Außenhandel. Die Geldpolitik hat einen bedeutenden Einfluss sowohl auf das tatsächliche als auch auf das potenzielle BIP.
\subsection{Internationale Verbindungen}
Staaten behalten ihre Außenhandelsströme immer genau im Auge. Einen besonders wichtigen Index stellen dabei die Nettoexporte dar, das heißt die numerische Differenz zwischen dem Wert der Exporte und dem der Importe. Wenn mehr exportiert als importiert wird, kommt es zu einem Außenhandelsüberschuss, während eine negative Nettoexportbilanz ein Außenhandelsdefizit darstellt.\\
Die Weltwirtschaft stellt ein dichtes Netz aus Handels- und Finanzbeziehungen zwischen den Ländern dar. Funktioniert das internationale Wirtschaftssystem gut, trägt es zu raschem Wirtschaftswachstum bei; wenn Handelssysteme dagegen zusammenbrechen, leiden weltweit Produktion und Einkommen darunter. Daher denken die Staaten sorgfältig über die Auswirkungen ihrer Handelspolitik und ihres internationalen Finanzmanagements auf ihre innenpolitischen Ziele Produktionsleistung, Beschäftigung und Preisstabilität nach. 
\subsection{Gesamtnachfrage und Gesamtangebot}
\subsubsection{Gesamtangebot}
Als Gesamtangebot bezeichnet man die Gesamtmenge aller Waren und Dienstleistungen, die die Unternehmen eines Landes in einem bestimmten Zeitraum zu produzieren und zu verkaufen gewillt sind.
\subsubsection{Gesamtnachfrage}
Die Gesamtnachfrage, also die Gesamtsumme der von den verschiedenen Wirtschaftssektoren während eines bestimmten Zeitraums freiwillig getätigtenAusgaben. Die Gesamtnachfrage (auch als AD bezeichnet) ist die Summe der Ausgaben der Konsumenten, Unternehmen und Regierungsorgane und hängt vom Preisniveau, der Geldpolitik, der Fiskalpolitik und weiteren Faktoren ab.
\subsubsection{Makroökonomisches Gleichgewicht}
Ein makroökonomisches Gleichgewicht ist eine Kombination aus Gesamtpreis und -menge, bei der alle Käufer und Verkäufer mit ihren Käufen, Verkäufen und den Preisen zufrieden sind. \\
\includegraphics[width=0.88\textwidth]{img/gesamt.jpg}



























\end{document}
